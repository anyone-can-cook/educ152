% Options for packages loaded elsewhere
\PassOptionsToPackage{unicode}{hyperref}
\PassOptionsToPackage{hyphens}{url}
\PassOptionsToPackage{dvipsnames,svgnames*,x11names*}{xcolor}
%
\documentclass[
  8pt,
  ignorenonframetext,
  dvipsnames]{beamer}
\usepackage{pgfpages}
\setbeamertemplate{caption}[numbered]
\setbeamertemplate{caption label separator}{: }
\setbeamercolor{caption name}{fg=normal text.fg}
\beamertemplatenavigationsymbolsempty
% Prevent slide breaks in the middle of a paragraph
\widowpenalties 1 10000
\raggedbottom
\setbeamertemplate{part page}{
  \centering
  \begin{beamercolorbox}[sep=16pt,center]{part title}
    \usebeamerfont{part title}\insertpart\par
  \end{beamercolorbox}
}
\setbeamertemplate{section page}{
  \centering
  \begin{beamercolorbox}[sep=12pt,center]{part title}
    \usebeamerfont{section title}\insertsection\par
  \end{beamercolorbox}
}
\setbeamertemplate{subsection page}{
  \centering
  \begin{beamercolorbox}[sep=8pt,center]{part title}
    \usebeamerfont{subsection title}\insertsubsection\par
  \end{beamercolorbox}
}
\AtBeginPart{
  \frame{\partpage}
}
\AtBeginSection{
  \ifbibliography
  \else
    \frame{\sectionpage}
  \fi
}
\AtBeginSubsection{
  \frame{\subsectionpage}
}
\usepackage{lmodern}
\usepackage{amssymb,amsmath}
\usepackage{ifxetex,ifluatex}
\ifnum 0\ifxetex 1\fi\ifluatex 1\fi=0 % if pdftex
  \usepackage[T1]{fontenc}
  \usepackage[utf8]{inputenc}
  \usepackage{textcomp} % provide euro and other symbols
\else % if luatex or xetex
  \usepackage{unicode-math}
  \defaultfontfeatures{Scale=MatchLowercase}
  \defaultfontfeatures[\rmfamily]{Ligatures=TeX,Scale=1}
\fi
% Use upquote if available, for straight quotes in verbatim environments
\IfFileExists{upquote.sty}{\usepackage{upquote}}{}
\IfFileExists{microtype.sty}{% use microtype if available
  \usepackage[]{microtype}
  \UseMicrotypeSet[protrusion]{basicmath} % disable protrusion for tt fonts
}{}
\makeatletter
\@ifundefined{KOMAClassName}{% if non-KOMA class
  \IfFileExists{parskip.sty}{%
    \usepackage{parskip}
  }{% else
    \setlength{\parindent}{0pt}
    \setlength{\parskip}{6pt plus 2pt minus 1pt}}
}{% if KOMA class
  \KOMAoptions{parskip=half}}
\makeatother
\usepackage{xcolor}
\IfFileExists{xurl.sty}{\usepackage{xurl}}{} % add URL line breaks if available
\IfFileExists{bookmark.sty}{\usepackage{bookmark}}{\usepackage{hyperref}}
\hypersetup{
  pdftitle={Introduction to Multivariate Regression \& Econometrics},
  pdfauthor={Lecture 12},
  colorlinks=true,
  linkcolor=Maroon,
  filecolor=Maroon,
  citecolor=Blue,
  urlcolor=blue,
  pdfcreator={LaTeX via pandoc}}
\urlstyle{same} % disable monospaced font for URLs
\newif\ifbibliography
\setlength{\emergencystretch}{3em} % prevent overfull lines
\providecommand{\tightlist}{%
  \setlength{\itemsep}{0pt}\setlength{\parskip}{0pt}}
\setcounter{secnumdepth}{-\maxdimen} % remove section numbering

%packages
\usepackage{graphicx}
\usepackage{rotating}
\usepackage{hyperref}

\usepackage{tikz} % used for text highlighting, amongst others
\usepackage{comment}

%title slide stuff
%\institute{Department of Education}
%\title{Managing and Manipulating Data Using R}

%
\setbeamertemplate{navigation symbols}{} % get rid of navigation icons:
\setbeamertemplate{footline}[page number]

%\setbeamertemplate{frametitle}{\thesection \hspace{0.2cm} \insertframetitle}
\setbeamertemplate{section in toc}[sections numbered]
%\setbeamertemplate{subsection in toc}[subsections numbered]
\setbeamertemplate{subsection in toc}{%
  \leavevmode\leftskip=3.2em\color{gray}\rlap{\hskip-2em\inserttocsectionnumber.\inserttocsubsectionnumber}\inserttocsubsection\par
}

%define colors
%\definecolor{uva_orange}{RGB}{216,141,42} % UVa orange (Rotunda orange)
\definecolor{mygray}{rgb}{0.95, 0.95, 0.95} % for highlighted text
	% grey is equal parts red, green, blue. higher values >> lighter grey
	%\definecolor{lightgraybo}{rgb}{0.83, 0.83, 0.83}

% new commands

%highlight text with very light grey
\newcommand*{\hlg}[1]{%
	\tikz[baseline=(X.base)] \node[rectangle, fill=mygray] (X) {#1};%
}
%, inner sep=0.3mm
%highlight text with very light grey and use font associated with code
\newcommand*{\hlgc}[1]{\texttt{\hlg{#1}}}

%modifying back ticks to add grey background
\let\OldTexttt\texttt
\renewcommand{\texttt}[1]{\OldTexttt{\hlg{#1}}}


\begin{comment}

% Font
\usepackage[defaultfam,light,tabular,lining]{montserrat}
\usepackage[T1]{fontenc}
\renewcommand*\oldstylenums[1]{{\fontfamily{Montserrat-TOsF}\selectfont #1}}

% Change color of boldface text to darkgray
\renewcommand{\textbf}[1]{{\color{darkgray}\bfseries\fontfamily{Montserrat-TOsF}#1}}

% Bullet points
\setbeamertemplate{itemize item}{\color{BlueViolet}$\circ$}
\setbeamertemplate{itemize subitem}{\color{BrickRed}$\triangleright$}
\setbeamertemplate{itemize subsubitem}{$-$}

% Reduce space before lists
%\addtobeamertemplate{itemize/enumerate body begin}{}{\vspace*{-8pt}}

\let\olditem\item
\renewcommand{\item}{%
  \olditem\vspace{4pt}
}

% decreasing space before and after level-2 bullet block
%\addtobeamertemplate{itemize/enumerate subbody begin}{}{\vspace*{-3pt}}
%\addtobeamertemplate{itemize/enumerate subbody end}{}{\vspace*{-3pt}}

% decreasing space before and after level-3 bullet block
%\addtobeamertemplate{itemize/enumerate subsubbody begin}{}{\vspace*{-2pt}}
%\addtobeamertemplate{itemize/enumerate subsubbody end}{}{\vspace*{-2pt}}

%Section numbering
\setbeamertemplate{section page}{%
    \begingroup
        \begin{beamercolorbox}[sep=10pt,center,rounded=true,shadow=true]{section title}
        \usebeamerfont{section title}\thesection~\insertsection\par
        \end{beamercolorbox}
    \endgroup
}

\setbeamertemplate{subsection page}{%
    \begingroup
        \begin{beamercolorbox}[sep=6pt,center,rounded=true,shadow=true]{subsection title}
        \usebeamerfont{subsection title}\thesection.\thesubsection~\insertsubsection\par
        \end{beamercolorbox}
    \endgroup
}

\end{comment}

\title{Introduction to Multivariate Regression \& Econometrics}
\subtitle{HED 612}
\author{Lecture 12}
\date{}

\begin{document}
\frame{\titlepage}

\begin{frame}
  \tableofcontents[hideallsubsections]
\end{frame}
\begin{frame}[fragile]{Prepare for class}
\protect\hypertarget{prepare-for-class}{}

We'll be using the ELS Data today; no need to re-download!

\medskip

\begin{enumerate}
\tightlist
\item
  Download the Lecture 12 PDF and R files for this week

  \begin{itemize}
  \tightlist
  \item
    Place all files in HED612\_SP21
    \textgreater\textgreater\textgreater{} lectures
    \textgreater\textgreater\textgreater{} lecture12
  \end{itemize}
\item
  Open the RProject (should be in your main HED612\_SP21 folder)
\item
  Once the RStudio window opens, open the Lecture 12 R script by
  clicking on:

  \begin{itemize}
  \tightlist
  \item
    file \textgreater\textgreater\textgreater{} open file\ldots{}
    \textgreater\textgreater\textgreater{} {[}navigate to lecture 12
    folder{]} \textgreater\textgreater\textgreater{} lecture12\_v2.R
  \end{itemize}
\item
  \textbf{Install the ``stargazer'' package in lecture12\_v2.R}

  \begin{itemize}
  \tightlist
  \item
    If R prompts you to install \texttt{stargazer} and ``its
    dependencies'', go ahead and click install
  \item
    If R doesn't prompt you, install via line 8
    \texttt{install.packages("stargazer")}
  \end{itemize}
\end{enumerate}

\end{frame}

\begin{frame}{Finishing the Semester\ldots{}}
\protect\hypertarget{finishing-the-semester}{}

\begin{itemize}
\tightlist
\item
  Today, 4/14

  \begin{itemize}
  \tightlist
  \item
    Linear Probability Models for 0/1 DV

    \begin{itemize}
    \tightlist
    \item
      Exporting regression ``nearly-publication'' ready tables!
    \end{itemize}
  \item
    Mini lesson on what each section of final project should accomplish!
  \item
    Finish Reviewing Powers(2004)
  \end{itemize}
\end{itemize}

\medskip

\begin{itemize}
\tightlist
\item
  4/21 {[}No Class/Reading Day{]}

  \begin{itemize}
  \tightlist
  \item
    Reading for 4/28

    \begin{itemize}
    \tightlist
    \item
      Klasik, D., Blagg, K., \& Pekor, Z. (2018). Out of the Education
      Desert: How Limited Local College Options are Associated with
      Inequity in Postsecondary Opportunities. Social Sciences, 7(9),
      2018.
    \end{itemize}
  \end{itemize}
\end{itemize}

\medskip

\begin{itemize}
\tightlist
\item
  4/28

  \begin{itemize}
  \tightlist
  \item
    Reviewing Empirical Research:

    \begin{itemize}
    \tightlist
    \item
      Klasik, D., Blagg, K., \& Pekor, Z. (2018). Out of the Education
      Desert: How Limited Local College Options are Associated with
      Inequity in Postsecondary Opportunities. Social Sciences, 7(9),
      2018.\\
    \end{itemize}
  \item
    Introduction to non-linear functions {[}we'll get as far as we
    can\ldots{]}
  \end{itemize}
\end{itemize}

\medskip

\begin{itemize}
\tightlist
\item
  Last class session, 5/5

  \begin{itemize}
  \tightlist
  \item
    Student presentations!

    \begin{itemize}
    \tightlist
    \item
      Don't stress these; don't need to be perfect!
    \item
      Just an opportunity to learn from each other and get some practice
      ``presenting''
    \end{itemize}
  \item
    \textbf{Final paper is due on our ``final exam day/time'': 5/7/21 by
    5:30pm}
  \end{itemize}
\end{itemize}

\end{frame}

\hypertarget{linear-probability-model}{%
\section{Linear Probability Model}\label{linear-probability-model}}

\begin{frame}{Linear Probability Model}
\protect\hypertarget{linear-probability-model-1}{}

\begin{itemize}
\tightlist
\item
  Binary Variables (i.e., dummies, indicators) as dependent variables
  are very common in education research!

  \begin{itemize}
  \tightlist
  \item
    Y = Retention (0=dropped out, 1= persisted)
  \item
    Y = Graduation (0= did not graduate, 1= graduated)
  \item
    Y = Pass/Fail (0=Failed, Passed=1)
  \end{itemize}
\end{itemize}

\medskip

\begin{itemize}
\tightlist
\item
  Regression models with a binary dependent variable attempt to
  interpret the effect of X on the \emph{probability} of ``success''
  (Y=1)

  \begin{itemize}
  \tightlist
  \item
    Or in some cases the probability of ``failure''
  \end{itemize}
\end{itemize}

\medskip

\begin{itemize}
\tightlist
\item
  Most social science disciplines model binary dependent variables via
  non-linear regression models

  \begin{itemize}
  \tightlist
  \item
    logistic regression {[}will cover in HED 649 Spring 2022{]}
  \item
    but interpretation can be difficult because its measured via odds
    ratios\\
  \end{itemize}
\end{itemize}

\medskip

\begin{itemize}
\tightlist
\item
  Econometrics models binary dependent variables via \textbf{linear
  probability model}

  \begin{itemize}
  \tightlist
  \item
    Population parameters can be estimated via OLS!
  \item
    Simple to estimate and interpret!
  \item
    Only ``tool'' that doesn't carry over? \(R^2\); but program
    evaluation is less concerned with model fit than hypothesis testing
    about the population parameter \(\beta_1\)
  \item
    It can ONLY be used for binary, categorical dependent variables

    \begin{itemize}
    \tightlist
    \item
      If more than two categories, then we have to use different models
      covered in HED 649
    \end{itemize}
  \end{itemize}
\end{itemize}

\end{frame}

\begin{frame}[fragile]{Linear Probability Model, with Continuous X}
\protect\hypertarget{linear-probability-model-with-continuous-x}{}

\begin{itemize}
\tightlist
\item
  RQ: What is the effect of hours spent on homework on the probability
  of attending college?
\item
  \textbf{Pop Reg Model: \(Yi = \beta_0 + \beta_1X_{1i} + ui\)}

  \begin{itemize}
  \tightlist
  \item
    Y = \texttt{college}(1= attended college, 0= did not attend college)
  \item
    X = average hours spent on homework per week
  \end{itemize}
\item
  Run in R
\item
  \textbf{OLS Prediction Line}

  \begin{itemize}
  \tightlist
  \item
    w/o estimates: \(Yi = \hat{\beta_0} + \hat{\beta_1}X_{1i}\)
  \item
    w estimates: \(Yi = -0.083759 + 0.019575*X_{1i}\)
  \end{itemize}
\item
  \textbf{Interpretation of \(\hat{\beta_1}\)}

  \begin{itemize}
  \tightlist
  \item
    General: On average, a one-unit increase in X is associated with a
    \(\hat{\beta_1}*100\) percentage point change in the probability of
    Y=1
  \item
    On average, a one-hour increase in average homework hours spent per
    week is associated with a \textasciitilde2 \((~0.02*100)\)
    percentage-point increase in the probability of going to college.
  \end{itemize}
\item
  \textbf{Interpretation of \(\hat{\beta_1}\)} when you SCALE X

  \begin{itemize}
  \tightlist
  \item
    General: On average, a N-unit increase in X is associated with a
    ((N-unit\(*\hat{\beta_1}\))*100) percentage point change in the
    probability of Y=1
  \item
    On average, a five-hour increase in average homework hours spent per
    week is associated with a 10 percentage-point increase
    \(((5*0.02)*100)\) in the probability of going to college.
  \end{itemize}
\end{itemize}

\end{frame}

\begin{frame}[fragile]{Linear Probability Model, with Categorical X}
\protect\hypertarget{linear-probability-model-with-categorical-x}{}

\begin{itemize}
\tightlist
\item
  RQ: What is the effect of high school extracurricular participation on
  the probability of attending college?
\item
  \textbf{Pop Reg Model:
  \(Yi = \beta_0 + \beta_1X_{1i} + \beta_2X_{2i} + ui\)}

  \begin{itemize}
  \tightlist
  \item
    Y = \texttt{college} (1= attended college, 0= did not attend
    college)
  \item
    X1 = \texttt{excurr} (1= participated in extracurriculars, 0= did
    not participate {[}reference group{]})
  \item
    X2 = average hours spent on homework per week
  \end{itemize}
\item
  Run in R
\item
  \textbf{OLS Prediction Line}

  \begin{itemize}
  \tightlist
  \item
    w/o estimates:
    \(Yi = \hat{\beta_0} + \hat{\beta_1}X_{1i} + \hat{\beta_2}X_{2i}\)
  \item
    w estimates: \(Yi = -0.227656 + 0.350495*X_{1i} + 0.016144*X_{2i}\)
  \end{itemize}
\item
  \textbf{Interpretation of \(\hat{\beta_1}\)}

  \begin{itemize}
  \tightlist
  \item
    General: On average, being in the ``non-reference group'' as opposed
    to the ``reference group'' is associated with a
    100*\(\hat{\beta_1}\) percentage point change in the probability of
    Y=1
  \item
    On average, participating in extracurriculars as opposed to not
    participating in extracurriculars is associated with a 35
    (0.350495*100) percentage point increase in the probability of
    attending college, holding hours spent on homework constant
  \end{itemize}
\end{itemize}

\end{frame}

\hypertarget{creating-nearly-publication-quality-regression-tables}{%
\section{Creating (nearly!) Publication Quality Regression
Tables}\label{creating-nearly-publication-quality-regression-tables}}

\begin{frame}[fragile]{Stargazer Library}
\protect\hypertarget{stargazer-library}{}

\begin{itemize}
\tightlist
\item
  Need to install first via \texttt{install.packages("stargazer")}
\item
  Package used to create publication ready tables
\item
  Some resourses

  \begin{itemize}
  \tightlist
  \item
    \href{https://cran.r-project.org/web/packages/stargazer/stargazer.pdf}{CRAN
    Package Documentation}
  \item
    \href{https://cran.r-project.org/web/packages/stargazer/vignettes/stargazer.pdf}{CRAN
    Package Vignettes}
  \item
    \href{https://www.princeton.edu/~otorres/NiceOutputR.pdf}{Helpful
    Presentation \& Summary}
  \item
    Google! Google! Google!
  \end{itemize}
\item
  Show in R!
\end{itemize}

\end{frame}

\hypertarget{final-projects}{%
\section{Final Projects}\label{final-projects}}

\begin{frame}{Approach to final projects\ldots{}}
\protect\hypertarget{approach-to-final-projects}{}

\begin{itemize}
\tightlist
\item
  I'm super chill about final project! :)

  \begin{itemize}
  \tightlist
  \item
    No pressure opportunity to learn and ``de-mystify'' the quant
    research process!
  \item
    I know we're all running on fumes\ldots{}
  \item
    I just want you to get what you can out of the assignment given
    space/time/energy you have!
  \end{itemize}
\item
  ``Presentations'' on 5/5

  \begin{itemize}
  \tightlist
  \item
    Do not need to be polished presentations
  \item
    Just a way to share with the class what you did
  \item
    We will learn from the process of putting together our own project
    and from eachother's work too!
  \item
    No requirements in terms of ``structure'': power point; share screen
    of your paper draft; run regression in RStudio for us, etc.
  \end{itemize}
\end{itemize}

\end{frame}

\begin{frame}{No More Problem Sets, Working Towards Final Projects}
\protect\hypertarget{no-more-problem-sets-working-towards-final-projects}{}

\begin{itemize}
\tightlist
\item
  By 4/21:

  \begin{itemize}
  \tightlist
  \item
    Finalize RQ
  \item
    Identify dataset, clean DV/IV\\
  \item
    Run bivariate regression/Interpret \(\hat{\beta_1}\)
  \item
    Identify 3-5 control variables that satisfy OVB

    \begin{itemize}
    \tightlist
    \item
      Clean control variable(s) if available in your dataset (\& if you
      have the time)
    \item
      Run multivariate regression/Interpret \(\hat{\beta_1}\)
    \end{itemize}
  \end{itemize}
\item
  By 4/28

  \begin{itemize}
  \tightlist
  \item
    Write up methods section and results\\
  \end{itemize}
\item
  By 5/5

  \begin{itemize}
  \tightlist
  \item
    Read and summarize 5 empirical articles that belong in your
    literature review
  \item
    Presentation does not need to be ``polished''; just sharing your RQ
    and results!
  \end{itemize}
\end{itemize}

\end{frame}

\begin{frame}{Final Project: Introduction}
\protect\hypertarget{final-project-introduction}{}

\begin{itemize}
\tightlist
\item
  An introduction is (in my humble opinion) the hardest section to
  write! Why?
\item
  It sets the tone for the entire study!
\item
  An introduction should:

  \begin{itemize}
  \tightlist
  \item
    Convince the reader of the significance and contributions of the
    study via a ``hook'' (i.e., Who should care about this study? and
    Why?)
  \item
    Present the research question
  \item
    Summarize what the study does
  \end{itemize}
\end{itemize}

\medskip

\begin{itemize}
\tightlist
\item
  \textbf{Your final paper's introduction should (at least) present the
  research question and summarize the study}
\end{itemize}

\medskip

\begin{itemize}
\tightlist
\item
  If you want try to ``hook'' the reader, here are some strategies:

  \begin{itemize}
  \tightlist
  \item
    Actionable research: You can take results and \emph{directly} make
    recommendations for changes to policy/practice
  \item
    Interesting research: New analyses on an important topic, resolving
    scholarly/public debates, introduces new ways to looking at old
    problems, has implications for theory
  \end{itemize}
\end{itemize}

\end{frame}

\begin{frame}{Final Project: Sometimes you may have a ``background''
section}
\protect\hypertarget{final-project-sometimes-you-may-have-a-background-section}{}

\begin{itemize}
\tightlist
\item
  Policy/Background sections are very common in econometrics research.

  \begin{itemize}
  \tightlist
  \item
    Because econometrics research usually focuses on understanding the
    effect of some program/ policy /intervention etc.
  \item
    The background section introduces the reader to the history and
    specific details of the program, policy, intervention being studied

    \begin{itemize}
    \tightlist
    \item
      Powers 2004 summarized all the legal background and precedence of
      the Williams v. State of CA case in this background section
      (p.766)
    \end{itemize}
  \end{itemize}
\end{itemize}

\medskip

\begin{itemize}
\tightlist
\item
  For example, if you are studying the effect of participating in Upward
  Bound, Gear Up, Head Start, College Prep Programs etc on some
  outcome\ldots.

  \begin{itemize}
  \tightlist
  \item
    Assume the reader has NO KNOWLEDGE of the program
  \item
    What is the program?
  \item
    What are the goals of the program?
  \item
    When was it created and for who?
  \item
    How has the program changed over time (i.e., eligibility,
    curriculum, financial support from government, etc.)
  \end{itemize}
\end{itemize}

\medskip  \textbf{If your independent variable of interest is a program,
policy, intervention, your final paper should include a background
section describing the program, policy, or intervention}

\end{frame}

\begin{frame}{Final Project: Literature Review}
\protect\hypertarget{final-project-literature-review}{}

\begin{itemize}
\tightlist
\item
  Literature reviews should accomplish the following main tasks:

  \begin{itemize}
  \tightlist
  \item
    Reviewing previous literature in your study's area
  \item
    Identifying the ``gap'' in previous literature
  \item
    Addressing how your study fills that ``gap''
  \end{itemize}
\item
  Literature reviews take a lot of time\ldots{}

  \begin{itemize}
  \tightlist
  \item
    A lot of reading; but also really fun!
  \end{itemize}
\item
  You of course won't have the time to write out a developed literature
  review and I don't expect you to!
\end{itemize}

\medskip

\textbf{For your final paper, I expect you to identify and briefly
summarize 5 empirical, peer-reviewed articles that belong in a fully
developed literature review. Explain WHY these belong in your literature
review (i.e., what are their contributions)}

\end{frame}

\begin{frame}{Final Projects: Methods and Data}
\protect\hypertarget{final-projects-methods-and-data}{}

Methods sections are straightforward but technical\ldots{}

\medskip

\textbf{For your final paper, the Methods Section should have the
following sections:}

\begin{itemize}
\tightlist
\item
  \textbf{Empirical Strategy}

  \begin{itemize}
  \tightlist
  \item
    Your aim is to analyze the effect of X on Y\ldots{}
  \item
    The gold standard of econometrics research is a randomized control
    experiment; explain how YOUR independent variable of interest would
    be hypothetically randomized/why it's not possible (e.g., financial
    aid is awarded based on need, Upward Bound was created to serve
    low-income students, etc)
  \item
    Thus, you are working with observational data and trying to get as
    close as possible to a causal effect (all you can do is get as close
    as possible to isolating the relationship of X on Y; but you can
    claim a causal relationship) by using control variables that would
    otherwise result in omitted variable bias (make sure you define
    OVB!)
  \item
    Cite Stock and Watson textbook; Cite Cellini (2007)
  \end{itemize}
\item
  \textbf{Data and Variables}

  \begin{itemize}
  \tightlist
  \item
    What data are you using? If NCES Survey, provide some details about
    the survey, the sample, years conducted, etc.
  \item
    Provide detailed information about your independent variable of
    interest (what was the survey question asked, how did you deal with
    missing values, did you transform this variable in any other way)
  \item
    Provide detailed information about your dependent variable (what was
    the survey question asked, how did you deal with missing values, did
    you transform this variable in any other way)
  \item
    Identify 3-5 control variables. Explain WHY they should be included
    in the model (i.e., how do they satisfy both conditions of OVB).
  \end{itemize}
\end{itemize}

\end{frame}

\begin{frame}{Final Projects: Methods and Data}
\protect\hypertarget{final-projects-methods-and-data-1}{}

\textbf{Analytical Strategy}

\begin{itemize}
\tightlist
\item
  Because your DV is continuous, you will run an Ordinary Least Squares
  Linear regression model

  \begin{itemize}
  \tightlist
  \item
    {[}or{]} Because your DV is binary, you will run a Linear
    Probability regression model
  \end{itemize}
\item
  Write out the population regression model and label what every piece
  of the equation represents.
\item
  Specify that given your RQ, your focus is on analyzing \(\beta_1\)
  because it represents the relationship between your X and Y

  \begin{itemize}
  \tightlist
  \item
    The magnitude of the coefficient shows the magnitude of the
    relationship:

    \begin{itemize}
    \tightlist
    \item
      the average effect on Y for a one-unit increase in X (continuous
      X, continuous Y)
    \item
      the average change in Y for the non-reference group in comparison
      to reference group (categorical X, continuous Y)
    \item
      the change in probability for going from Y=0 to Y=1 for a one unit
      increase in X (continuous X, binary Y)
    \item
      the change in probability for going from Y=0 to Y=1 for the
      non-reference group in comparison to reference group (categorical
      X, binary Y)
    \end{itemize}
  \item
    The p-value of the hypothesis test on the coefficient shows the
    statistical significance of the relationship.

    \begin{itemize}
    \tightlist
    \item
      State the Null and Alternative Hypotheses!
    \end{itemize}
  \end{itemize}
\end{itemize}

\medskip

A great example of how to write this section that you can emulate
{[}posted on D2L{]}:

\begin{itemize}
\tightlist
\item
  Nolan L. Cabrera, Jeffrey F. Milem, Ozan Jaquette, \& Roland W. Marx.
  (2014). Missing the (Student Achievement) Forest for All the
  (Political) Trees: Empiricism and the Mexican American Studies
  Controversy in Tucson. American Educational Research Journal, 51(6),
  1084--1118. \url{https://doi.org/10.3102/0002831214553705}
\end{itemize}

\end{frame}

\begin{frame}{Final Projects: Findings}
\protect\hypertarget{final-projects-findings}{}

This section will actually be very short\ldots{}

\begin{itemize}
\tightlist
\item
  Start by describing the descriptive statistics of your indepedent and
  dependent variable (mean, standard deviation, min, and max)
\item
  Interpret \(\hat{\beta_1}\)

  \begin{itemize}
  \tightlist
  \item
    In words and state the statistical significance
  \end{itemize}
\end{itemize}

\medskip

Create and include two tables to show the results:

\begin{itemize}
\tightlist
\item
  Table 1 should include the mean, standard deviation, min, max (counts
  and \% for categorical variables) for your independent variable of
  interest and dependent variable
\item
  Table 2 should include results of the bivariate regression and
  multivariate regression models
\end{itemize}

\end{frame}

\hypertarget{min-break}{%
\section{10 Min Break}\label{min-break}}

\hypertarget{learning-to-read-quantitative-empirical-work}{%
\section{Learning to Read Quantitative Empirical
Work}\label{learning-to-read-quantitative-empirical-work}}

\begin{frame}{Powers (2004): Overall RQ, Analytical and Writting
Approach!}
\protect\hypertarget{powers-2004-overall-rq-analytical-and-writting-approach}{}

\begin{itemize}
\tightlist
\item
  \textbf{Williams v. State of California}

  \begin{itemize}
  \tightlist
  \item
    Class action lawsuit against the State of California in 2000; public
    school students represented by a coalition of law firms, civil
    rights organizations
  \item
    Argued that if schools and students are judged on the basis of their
    test scores (high stakes accountability policies), then students and
    schools should be provided with equal access to school-related
    resources needed for academic success
  \item
    Specifically addressed need for qualified teachers,
    sufficient/up-to-date textbooks, and adequate/safe facilities
  \end{itemize}
\item
  \textbf{Old legal approach to school financing}: focused on levels and
  formulas of spending

  \begin{itemize}
  \tightlist
  \item
    Didn't work because the state has minimal oversight of local
    educational control/``property taxes''; caps were eliminated via
    voter overrides; and the small proportion of state funding was given
    ``equally''
  \end{itemize}
\item
  \textbf{New legal approach to school financing}: \emph{Williams v.
  State of California} reframed the issue of school financing around the
  \emph{conditions} of education rather than funding formulas!

  \begin{itemize}
  \tightlist
  \item
    Worked because the state is responsible for ensuring resources are
    used to produce desirable outcomes
  \end{itemize}
\item
  State of California argued that there was little empirical support
  that increased spending on resources identified in the case would
  increase student achievement

  \begin{itemize}
  \tightlist
  \item
    \textbf{RQ}: What is the relationship(s) between school and district
    characteristics on school's academic performance?
  \end{itemize}
\end{itemize}

\end{frame}

\begin{frame}{Powers (2004): Overall RQ, Analytical and Writting
Approach!}
\protect\hypertarget{powers-2004-overall-rq-analytical-and-writting-approach-1}{}

\begin{itemize}
\tightlist
\item
  \textbf{Empirical Strategy}

  \begin{itemize}
  \tightlist
  \item
    Powers is not trying to establish a causal effect of school/resource
    characteristics on API score explicitly (impossible given the
    observational data)
  \item
    But she is trying to isolate the Williams Case variables
    ``relationship'' on schools' academic performance by controlling for
    other factors that may be driving variation in schools' academic
    performance!

    \begin{itemize}
    \tightlist
    \item
      See on See pg.766 for rationale in including control variables
    \item
      ``To ensure the findings related to Williams variables are robust,
      that is, they are not systematically related to student, school,
      and district characteristics''
    \end{itemize}
  \item
    But still analyzing ``descriptive relationships'', not ``causal
    relationships''
  \end{itemize}
\item
  \textbf{Literature Review Strategy}

  \begin{itemize}
  \tightlist
  \item
    HOW you format your literature review depends on your RQ and what is
    substantively important
  \end{itemize}

  \begin{enumerate}
  [(1)]
  \tightlist
  \item
    Focus on reviewing scholarship about your X (most econometrics
    studies do this)

    \begin{itemize}
    \tightlist
    \item
      Ex: What is the effect of receiving a Pell Grant on achievement?;
      review scholarship on the effect of Pell grants on all student
      outcomes (GPA, graduation, labor market, etc.)
    \end{itemize}
  \item
    Focus on reviewing scholarship about your Y (many
    prediction/descriptive studies do this)

    \begin{itemize}
    \tightlist
    \item
      EX: What is the (predicted) probability of a student persisting
      into their 2nd year of college?; review scholarship on what we
      know about persistence (more likely to persist if you're involved
      on campus, live on campus, etc.)
    \end{itemize}
  \item
    Focus on reviewing scholarship that specifically speaks to the
    \emph{relationship} between X and Y

    \begin{itemize}
    \tightlist
    \item
      Ex: POWERS (2004, p.772): What do we know about the relationship
      between school resources and academic achievement
    \end{itemize}
  \end{enumerate}
\end{itemize}

\end{frame}

\begin{frame}{Powers (2004)}
\protect\hypertarget{powers-2004}{}

\begin{itemize}
\tightlist
\item
  \textbf{Data}: California DOE Data, school-level; Census Data; Federal
  School Data

  \begin{itemize}
  \tightlist
  \item
    Dependent variable(s): ?
  \item
    Independent variable(s) of interest: ?

    \begin{itemize}
    \tightlist
    \item
      Hint there are three groupings and they are related to the
      Williams Case
    \end{itemize}
  \item
    Control Variables: ?

    \begin{itemize}
    \tightlist
    \item
      Hint there are two groupings!
    \item
      Some control variables left out due to ``collinearity''
    \item
      \textbf{Collinearity}: When two or more of your independent
      variables are highly correlated; so correlated that adding just
      one of them explains both of their ``effect'' on the dependent
      variable. In other words including BOTH independent variables that
      are highly correlated is redundant and takes up statistical power
    \item
      R will automatically drop variables if there is collinearity!
    \end{itemize}
  \end{itemize}
\item
  \textbf{Model}: OLS Linear Regression

  \begin{itemize}
  \tightlist
  \item
    Powers doesn't write out the equation for the study!
  \item
    Let's write it ourselves! {[}at least for the three groups of
    variables for Williams Case{]}
  \end{itemize}
\end{itemize}

\end{frame}

\begin{frame}{Powers (2004)}
\protect\hypertarget{powers-2004-1}{}

\begin{itemize}
\tightlist
\item
  Table 2: Nested Regression Models Using 1999 API Index as DV, pg. 781

  \begin{itemize}
  \tightlist
  \item
    The most ``standard way'' to format regression results
  \item
    You show the ``progression'' towards your final regression model via
    multiple models

    \begin{itemize}
    \tightlist
    \item
      Intercept-only model (no independent variables) {[}Powers didnt
      show this{]}
    \item
      Model 1 can have your independent variable of interest (standard
      for program eval) or some variables (Powers did student
      characteristics)
    \item
      Models 2+\ldots{} shows the addition of control variables
    \end{itemize}
  \item
    Shows beta coefficients, standard errors, and significant levels!
  \end{itemize}
\item
  Does show model fit statistics!

  \begin{itemize}
  \tightlist
  \item
    Powers is not writing from a program evaluation standpoint
  \item
    Rather she's trying to show how variables from Williams Case explain
    (predict!) API scores
  \item
    In this case, measures of model fit are important
  \item
    \(R^2\) increases from Model 1 to Model 2

    \begin{itemize}
    \tightlist
    \item
      Change in \(R^2\) doesn't mean that Williams case IVs add
      ``little'' explanation; see pg 782!
    \end{itemize}
  \item
    In text, she mentions overall F-test is significant from Model 1 to
    Model 2
  \item
    F-test is related to \(R^2\)

    \begin{itemize}
    \tightlist
    \item
      \(H_0\): model with no independent variables (or limited IVs) fits
      the data just as well as model with full variables
    \item
      \(H_a\): model with more variables fits the data better than model
      with no IVs
    \end{itemize}
  \end{itemize}
\end{itemize}

\end{frame}

\begin{frame}{Powers (2004)}
\protect\hypertarget{powers-2004-2}{}

\begin{itemize}
\tightlist
\item
  Interpretation for Table 2 for Williams Case Variables
\item
  In text, she hardly interprets beta coefficients beyond ``significance
  and direction''

  \begin{itemize}
  \tightlist
  \item
    Only sometimes mentions the magnitude of the coefficient
  \item
    This approach is common in ``descriptive research''
  \end{itemize}
\item
  \textbf{Teacher training} (reference group is fully credentialed
  teachers)

  \begin{itemize}
  \tightlist
  \item
    \textbf{Emergency Credentialed \(\hat{\beta}\) Coefficient: -1.12***
    SE= 0.09}

    \begin{itemize}
    \tightlist
    \item
      Interpretation: ``On average, one-percentage-point increase in a
      school teaching staff's proportion of emergency credentialed
      teachers (as opposed to fully credentialed teachers) is associated
      with a 1.12 point decrease in API score, holding all covariates
      constant''
    \end{itemize}
  \end{itemize}
\item
  \textbf{Teacher Experience}

  \begin{itemize}
  \tightlist
  \item
    \textbf{Years teaching \(\hat{\beta}\) Coefficient: .80** SE= 0.26}

    \begin{itemize}
    \tightlist
    \item
      Interpretation: ``A one year increase in a school teaching staff's
      average years of experience, on average, is associated with a 0.80
      point increase in API score, holding all covariates constant''\\
    \end{itemize}
  \end{itemize}
\item
  \textbf{Teacher Education} (reference group is less than M.A)

  \begin{itemize}
  \tightlist
  \item
    \textbf{Greater than MA \(\hat{\beta}\) Coefficient: .42*** SE=
    0.06}

    \begin{itemize}
    \tightlist
    \item
      Interpretation: ``On average, a one-percentage-point increase in a
      school teaching staff's proportion of teachers with greater than
      MA degree (as opposed to lower than MA teachers) is associated
      with a 0.42 point increase in API score, holding all covariates
      constant''
    \end{itemize}
  \end{itemize}
\item
  \textbf{School Calendar} (reference group is traditional year)

  \begin{itemize}
  \tightlist
  \item
    \textbf{Concept 6 \(\hat{\beta}\) Coefficient: -34.46*** SE= 4.52}

    \begin{itemize}
    \tightlist
    \item
      Interpretation: ``On average, being a school with a Concept 6
      Calendar as opposed to a Traditional Calendar is associated with a
      34.46 point decrease in API score, holding all covariates
      constant''
    \end{itemize}
  \end{itemize}
\item
  \textbf{Textbooks}

  \begin{itemize}
  \tightlist
  \item
    \textbf{Per-pupil textbook expenditures \(\hat{\beta}\) Coefficient:
    .11*** SE=0.003}

    \begin{itemize}
    \tightlist
    \item
      Interpretation: ``A \$1 increase in per-pupil expenditures for
      textbooks is associated with a 0.1 point increase in API score,
      holding all covariates constant''\\
    \item
      Interpretation: ``A \$10 increase in per-pupil expenditures for
      textbooks, on average, is associated with a 1 point increase in
      API score, holding all covariates constant''\\
    \item
      Interpretation: ``A \$100 increase in per-pupil expenditures for
      textbooks, on average, is associated with a 10 point increase in
      API score, holding all covariates constant''
    \end{itemize}
  \end{itemize}
\end{itemize}

\end{frame}

\begin{frame}{Powers (2004)}
\protect\hypertarget{powers-2004-3}{}

Why do I like this piece of scholarship?

\begin{itemize}
\tightlist
\item
  The methods are simple!

  \begin{itemize}
  \tightlist
  \item
    Impactful research does not need to have complex methods!
  \item
    Study does not use any analyses/methods beyond what we have learned
    so far in an introductory regression class; yet it is published in
    the AERA flagship journal
  \end{itemize}
\item
  Example of how to deal with ``metrics'' that I fundamentally disagree
  with but are part of our educational reality

  \begin{itemize}
  \tightlist
  \item
    Metrics like SAT/ACT scores, standardized K-12 test, are ``firmly
    entrenched part of the political landscape'' that schools and
    students must navigate
  \item
    In many cases, they are institutionalized within federal, state, and
    university/school policies
  \item
    ``Aim is to make a strong argument for equity by marshaling the very
    data that dominate the political discourse'' (p.~765)
  \end{itemize}
\item
  Direct response to policies
\item
  Cool understanding of legal logic when it comes to education finance!
\end{itemize}

\end{frame}

\end{document}
