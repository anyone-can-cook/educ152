\PassOptionsToPackage{unicode=true}{hyperref} % options for packages loaded elsewhere
\PassOptionsToPackage{hyphens}{url}
\PassOptionsToPackage{dvipsnames,svgnames*,x11names*}{xcolor}
%
\documentclass[8pt,ignorenonframetext,dvipsnames]{beamer}
\usepackage{pgfpages}
\setbeamertemplate{caption}[numbered]
\setbeamertemplate{caption label separator}{: }
\setbeamercolor{caption name}{fg=normal text.fg}
\beamertemplatenavigationsymbolsempty
% Prevent slide breaks in the middle of a paragraph:
\widowpenalties 1 10000
\raggedbottom
\setbeamertemplate{part page}{
\centering
\begin{beamercolorbox}[sep=16pt,center]{part title}
  \usebeamerfont{part title}\insertpart\par
\end{beamercolorbox}
}
\setbeamertemplate{section page}{
\centering
\begin{beamercolorbox}[sep=12pt,center]{part title}
  \usebeamerfont{section title}\insertsection\par
\end{beamercolorbox}
}
\setbeamertemplate{subsection page}{
\centering
\begin{beamercolorbox}[sep=8pt,center]{part title}
  \usebeamerfont{subsection title}\insertsubsection\par
\end{beamercolorbox}
}
\AtBeginPart{
  \frame{\partpage}
}
\AtBeginSection{
  \ifbibliography
  \else
    \frame{\sectionpage}
  \fi
}
\AtBeginSubsection{
  \frame{\subsectionpage}
}
\usepackage{lmodern}
\usepackage{amssymb,amsmath}
\usepackage{ifxetex,ifluatex}
\usepackage{fixltx2e} % provides \textsubscript
\ifnum 0\ifxetex 1\fi\ifluatex 1\fi=0 % if pdftex
  \usepackage[T1]{fontenc}
  \usepackage[utf8]{inputenc}
  \usepackage{textcomp} % provides euro and other symbols
\else % if luatex or xelatex
  \usepackage{unicode-math}
  \defaultfontfeatures{Ligatures=TeX,Scale=MatchLowercase}
\fi
% use upquote if available, for straight quotes in verbatim environments
\IfFileExists{upquote.sty}{\usepackage{upquote}}{}
% use microtype if available
\IfFileExists{microtype.sty}{%
\usepackage[]{microtype}
\UseMicrotypeSet[protrusion]{basicmath} % disable protrusion for tt fonts
}{}
\IfFileExists{parskip.sty}{%
\usepackage{parskip}
}{% else
\setlength{\parindent}{0pt}
\setlength{\parskip}{6pt plus 2pt minus 1pt}
}
\usepackage{xcolor}
\usepackage{hyperref}
\hypersetup{
            pdftitle={Introduction to Multivariate Regression \& Program Evaluation},
            pdfauthor={Lecture 10},
            colorlinks=true,
            linkcolor=Maroon,
            filecolor=Maroon,
            citecolor=Blue,
            urlcolor=blue,
            breaklinks=true}
\urlstyle{same}  % don't use monospace font for urls
\newif\ifbibliography
\setlength{\emergencystretch}{3em}  % prevent overfull lines
\providecommand{\tightlist}{%
  \setlength{\itemsep}{0pt}\setlength{\parskip}{0pt}}
\setcounter{secnumdepth}{0}

% set default figure placement to htbp
\makeatletter
\def\fps@figure{htbp}
\makeatother


%packages
\usepackage{graphicx}
\usepackage{rotating}
\usepackage{hyperref}

\usepackage{tikz} % used for text highlighting, amongst others
\usepackage{comment}

%title slide stuff
%\institute{Department of Education}
%\title{Managing and Manipulating Data Using R}

%
\setbeamertemplate{navigation symbols}{} % get rid of navigation icons:
\setbeamertemplate{footline}[page number]

%\setbeamertemplate{frametitle}{\thesection \hspace{0.2cm} \insertframetitle}
\setbeamertemplate{section in toc}[sections numbered]
%\setbeamertemplate{subsection in toc}[subsections numbered]
\setbeamertemplate{subsection in toc}{%
  \leavevmode\leftskip=3.2em\color{gray}\rlap{\hskip-2em\inserttocsectionnumber.\inserttocsubsectionnumber}\inserttocsubsection\par
}

%define colors
%\definecolor{uva_orange}{RGB}{216,141,42} % UVa orange (Rotunda orange)
\definecolor{mygray}{rgb}{0.95, 0.95, 0.95} % for highlighted text
	% grey is equal parts red, green, blue. higher values >> lighter grey
	%\definecolor{lightgraybo}{rgb}{0.83, 0.83, 0.83}

% new commands

%highlight text with very light grey
\newcommand*{\hlg}[1]{%
	\tikz[baseline=(X.base)] \node[rectangle, fill=mygray] (X) {#1};%
}
%, inner sep=0.3mm
%highlight text with very light grey and use font associated with code
\newcommand*{\hlgc}[1]{\texttt{\hlg{#1}}}

%modifying back ticks to add grey background
\let\OldTexttt\texttt
\renewcommand{\texttt}[1]{\OldTexttt{\hlg{#1}}}


\begin{comment}

% Font
\usepackage[defaultfam,light,tabular,lining]{montserrat}
\usepackage[T1]{fontenc}
\renewcommand*\oldstylenums[1]{{\fontfamily{Montserrat-TOsF}\selectfont #1}}

% Change color of boldface text to darkgray
\renewcommand{\textbf}[1]{{\color{darkgray}\bfseries\fontfamily{Montserrat-TOsF}#1}}

% Bullet points
\setbeamertemplate{itemize item}{\color{BlueViolet}$\circ$}
\setbeamertemplate{itemize subitem}{\color{BrickRed}$\triangleright$}
\setbeamertemplate{itemize subsubitem}{$-$}

% Reduce space before lists
%\addtobeamertemplate{itemize/enumerate body begin}{}{\vspace*{-8pt}}

\let\olditem\item
\renewcommand{\item}{%
  \olditem\vspace{4pt}
}

% decreasing space before and after level-2 bullet block
%\addtobeamertemplate{itemize/enumerate subbody begin}{}{\vspace*{-3pt}}
%\addtobeamertemplate{itemize/enumerate subbody end}{}{\vspace*{-3pt}}

% decreasing space before and after level-3 bullet block
%\addtobeamertemplate{itemize/enumerate subsubbody begin}{}{\vspace*{-2pt}}
%\addtobeamertemplate{itemize/enumerate subsubbody end}{}{\vspace*{-2pt}}

%Section numbering
\setbeamertemplate{section page}{%
    \begingroup
        \begin{beamercolorbox}[sep=10pt,center,rounded=true,shadow=true]{section title}
        \usebeamerfont{section title}\thesection~\insertsection\par
        \end{beamercolorbox}
    \endgroup
}

\setbeamertemplate{subsection page}{%
    \begingroup
        \begin{beamercolorbox}[sep=6pt,center,rounded=true,shadow=true]{subsection title}
        \usebeamerfont{subsection title}\thesection.\thesubsection~\insertsubsection\par
        \end{beamercolorbox}
    \endgroup
}

\end{comment}

\title{Introduction to Multivariate Regression \& Program Evaluation}
\providecommand{\subtitle}[1]{}
\subtitle{HED 612}
\author{Lecture 10}
\date{}

\begin{document}
\frame{\titlepage}

\begin{frame}
\tableofcontents[hideallsubsections]
\end{frame}
\begin{frame}{Where are we going\ldots{}.}
\protect\hypertarget{where-are-we-going.}{}

\begin{itemize}
\tightlist
\item
  This Lecture

  \begin{itemize}
  \tightlist
  \item
    Intro to multivariate regression
  \item
    Intro to interpreting published multivariate regression results
  \item
    Reading for next lecture:

    \begin{itemize}
    \tightlist
    \item
      Cabrera, N. L., Milem, J. F., Jaquette, O., \& Marx, R. (2014).
      Missing the (student achievement) forest for all the (political)
      trees: Empiricism and the Mexican American Studies controversy in
      Tucson. American Educational Research Journal, 51(6), 1084-1118.
    \item
      Powers, J. M. (2004). High-Stakes Accountability and Equity: Using
      Evidence From California's Public Schools Accountability Act to
      Address the Issues in Williams v. State of California. American
      Educational Research Journal, 41(4), 763--795.
    \end{itemize}
  \item
    Homework \#10 posted!
  \end{itemize}
\item
  Next Lecture

  \begin{itemize}
  \tightlist
  \item
    Other OLS assumptions
  \item
    Graphing multivariate regression results
  \item
    Creating publication quality tables
  \end{itemize}
\item
  Next Next Lecture

  \begin{itemize}
  \tightlist
  \item
    Introduction to non-linear relationships between X and Y
  \item
    Mini lesson on what each section of manuscrupt should accomplish!
  \end{itemize}
\end{itemize}

\end{frame}

\hypertarget{introduction-to-multivariate-regression}{%
\section{Introduction to Multivariate
Regression}\label{introduction-to-multivariate-regression}}

\begin{frame}{Population Regression Model}
\protect\hypertarget{population-regression-model}{}

\begin{itemize}
\tightlist
\item
  Same as in ``simple'' (univariate) regression!
\item
  \(Y_i = \beta_0 + \beta_1X_{1i} + \beta_2X_{2i} +\) \ldots{}
  \(\beta_kX_{ki} + u_i\)
\item
  Where:

  \begin{itemize}
  \tightlist
  \item
    \(Y_i\) = observation i of dependent variable
  \item
    \(X_{1i}\) = observation i of the first regressor, \(X_1\)
  \item
    \(X_{2i}\) = observation i of the second regressor, \(X_2\)
  \item
    \(X_{ki}\) = observation i of the Kth regressor, \(X_k\)
  \item
    \(\beta_1\) = population average effect of Y for one-unit increase
    in \(X_1\)
  \item
    \(\beta_2\) = population average effect of Y for one-unit increase
    in \(X_2\)
  \item
    \(\beta_k\) = population average effect of Y for one-unit increase
    in \(X_k\)
  \item
    \(\beta_0\) = average value of Y when the value of all independent
    variables (\(X_1, X_2 ...X_k\)) are equal to zero
  \item
    \(u_i\) = all other variables that affect the valye of \(Y_i\) but
    are not included in the model
  \end{itemize}
\end{itemize}

\end{frame}

\begin{frame}{Things we do in multiple regression}
\protect\hypertarget{things-we-do-in-multiple-regression}{}

\begin{enumerate}
\tightlist
\item
  Estimation
\end{enumerate}

\begin{itemize}
\tightlist
\item
  Choose estimates for \(\beta_0, \beta_1, \beta_2, ... \beta_k\) by
  selecting those that minimuze the sum of squared errors (i.e., make
  the best prediction of Y), yielding an OLS line

  \begin{itemize}
  \tightlist
  \item
    \(\hat{Y_i} = \hat{\beta_0} + \hat{\beta_1} X_{1i} + \hat{\beta_2} X_{2i} + ... \hat{\beta_k} X_{ki}\)
  \end{itemize}
\end{itemize}

\begin{enumerate}
\setcounter{enumi}{1}
\tightlist
\item
  Measures of model fit (e.g., \(R^2\), SER)
\end{enumerate}

\begin{itemize}
\tightlist
\item
  But formulas change slightly to account for degrees of freedom!
\item
  Once you introduce multiple independent variables, use adjusted
  R-squared
\item
  Adjusted R-squared

  \begin{itemize}
  \tightlist
  \item
    Adjusted for the number of predictors in the model
  \item
    Every independent variable we add to the model will increase our
    ``normal'' R-squared; but doesn't necessarily mean it's a better
    fit!
  \item
    Adjusted R squared increases only if new variable improves the model
    more than would be expected by chance!
  \end{itemize}
\end{itemize}

\begin{enumerate}
\setcounter{enumi}{2}
\tightlist
\item
  Prediction
\end{enumerate}

\begin{itemize}
\tightlist
\item
  Once you estimate OLS regression line, we can calculate predicted
  values for observations with particular values of all independent
  variables

  \begin{itemize}
  \tightlist
  \item
    \(\hat{Y_i} = \hat{\beta_0} + \hat{\beta_1} X_{1i} + \hat{\beta_2} X_{2i} + ... \hat{\beta_k} X_{ki}\)
  \end{itemize}
\end{itemize}

\begin{enumerate}
\setcounter{enumi}{3}
\tightlist
\item
  Hypothesis testing and confidence intervals about \(\beta_1\)
\end{enumerate}

\begin{itemize}
\tightlist
\item
  Same as before but forumals for \(\hat{\beta_1}\) and
  \(SE(\hat{\beta_1})\) change slightly, but R calculates this for us!
\end{itemize}

\end{frame}

\begin{frame}{Conditional Independence Assumption}
\protect\hypertarget{conditional-independence-assumption}{}

\begin{itemize}
\tightlist
\item
  Assume students choose to participate in MAS
\item
  \(Y_i = \beta_0 + \beta_1X_{1i} + \beta_2X_{2i} + \beta_3X_{3i} + u_i\)
\item
  Where: Y=graduation, X1= 0/1 MAS, X2= previous academic achievement,
  X3= SES
\item
  \textbf{Conditional independence assumption}:

  \begin{itemize}
  \tightlist
  \item
    Once we include control variables, there are no omitted variables,
    Z, that satisfy \emph{both} of these two conditions:

    \begin{enumerate}
    [(1)]
    \tightlist
    \item
      Z affects value of Y \emph{and}
    \item
      Z has a relationship with X
    \end{enumerate}
  \end{itemize}
\item
  If the conditional independent assumption is true:

  \begin{itemize}
  \tightlist
  \item
    Once we include relevant control variables, there are no omitted
    variables that affect Y and have a relationship with X
  \item
    MAIN POINT: if we satisfy the conditional independence assumption
    through control variables, then multiple regression is just as good
    as randomized assignment experiment!
  \end{itemize}
\end{itemize}

\end{frame}

\begin{frame}{Multiavariate regression in Program Evaluation vs Social
Science}
\protect\hypertarget{multiavariate-regression-in-program-evaluation-vs-social-science}{}

\begin{itemize}
\tightlist
\item
  \(Y_i = \beta_0 + \beta_1X_{1i} + \beta_2X_{2i} + \beta_3X_{3i} + u_i\)
\item
  Program evaluation research or ``econometrics''

  \begin{itemize}
  \tightlist
  \item
    We are only interested in estimating \(\beta_1\) {[}the causal
    effect of X1 on Y{]}
  \item
    The only reason we include other variables in the model besides X1
    is to eliminate omitted variable bias
  \item
    Therefore, we include all control variables that satisfy \emph{both}
    conditions of omitted variable bias
  \item
    once we include control variables, and no other variables satisfy
    both conditions, then we satisfy the conditional independence
    assumption and we can estimate a causal effect!
  \end{itemize}
\item
  Traditional social science statistics {[}most of my research!{]}

  \begin{itemize}
  \tightlist
  \item
    Purpose of multiple regression is to add new variable to your model
    (e.g., \(X_3\)) to see the effect of \(X_3\) on Y
  \item
    Can lead to sloppy research if you're not careful!
  \item
    We ``throw'' everything and the kitchen sink into a model and see
    what's interesting!
  \end{itemize}
\end{itemize}

\end{frame}

\begin{frame}{Multivariate regression in R}
\protect\hypertarget{multivariate-regression-in-r}{}

\begin{itemize}
\tightlist
\item
  Research question: What is the effect of student teacher ratio on
  student reading test scores?
\item
  Simple regression

  \begin{itemize}
  \tightlist
  \item
    \(Y_i = \beta_0 + \beta_1X_{1i} + u_i\)
  \item
    Where: Y= reading test scores and \(X_1\) = student teacher ratio
  \item
    Interpretation of \(\hat{\beta_1}\): The average effect of a
    one-unit increase in \(X_1\) is associated with a \(\hat{\beta_1}\)
    change in Y
  \end{itemize}
\item
  Multivariate regression

  \begin{itemize}
  \tightlist
  \item
    \(Y_i = \beta_0 + \beta_1X_{1i} + \beta_2X_{2i} + u_i\)
  \item
    Where: Y= student test scores, \(X_1\) = student teacher ratio,
    \(X_2\) = \% ELL
  \item
    Interpretation of \(\hat{\beta_1}\): The average effect of a
    one-unit increase in \(X_1\) is associated with a \(\hat{\beta_1}\)
    change in Y, holding the value of \(X_2\) constant
  \end{itemize}
\end{itemize}

\end{frame}

\begin{frame}{What does ``holding constant'' mean?}
\protect\hypertarget{what-does-holding-constant-mean}{}

\begin{itemize}
\tightlist
\item
  RQ: What is the effect of student teacher ratio on reading test
  scores?

  \begin{itemize}
  \tightlist
  \item
    \(Y_i = \beta_0 + \beta_1X_{1i} + \beta_2X_{2i} + u_i\)
  \item
    Where: Y= student test scores, \(X_1\) = student teacher ratio,
    \(X_2\) = \% ELL
  \end{itemize}
\item
  Setup:

  \begin{itemize}
  \tightlist
  \item
    We think student test scores go down if there's a greater percentage
    of ELL students in the classroom

    \begin{itemize}
    \tightlist
    \item
      First condition of omitted variable bias ( Z affects Y)
    \end{itemize}
  \item
    We think there is a negative relationship between percentage of ELL
    students in the classroom and student-teacher ratio

    \begin{itemize}
    \tightlist
    \item
      Second condition of omitted variable bias ( Z has a relationship
      with X)
    \end{itemize}
  \end{itemize}
\item
  Problem:

  \begin{itemize}
  \tightlist
  \item
    We think student teacher ratio and percentage of ELL move together
  \item
    We want to know the relationship between reading scores and student
    teacher ratio when ``percent ELL'' is not allowed to move!
  \end{itemize}
\item
  ``Holding the value of \(X_2\) constant''

  \begin{itemize}
  \tightlist
  \item
    Means to estimate the relationship betwen \(X_1\) and Y when we
    don't allow the value of \(X_2\) to vary
  \item
    Said differenly: We analyze the relationship between student teacher
    ration (\(X_1\)) and reading test scores (Y) for applicants that
    have the same value of percent ELL (\(X_2\)) {[}calculus: partial
    derivatives!{]}
  \end{itemize}
\end{itemize}

\end{frame}

\begin{frame}{What does ``holding constant'' mean? Another
example\ldots{}.}
\protect\hypertarget{what-does-holding-constant-mean-another-example.}{}

\begin{itemize}
\tightlist
\item
  RQ: What is the relationship between years of education(X1) on
  income(Y), after controlling for years of work experience (X2)?
\item
  General interpretation of \(\hat{\beta_1}\):

  \begin{itemize}
  \tightlist
  \item
    The average effect of a one-unit increase in X1 is a
    \(\hat{\beta_1}\) unit increase in Y, holding the value of X2
    constant
  \end{itemize}
\item
  Interpretation of \(\hat{\beta_1}\), applied to example

  \begin{itemize}
  \tightlist
  \item
    The effect of having one additional year of education (X1) on income
    (Y), when we don't allow value of ``years of experience'' (X2) to
    change
  \item
    Maybe people w/ more education have fewer years of experience
  \end{itemize}
\item
  Said differently: analyze the effect of increasing years of education
  on income for people who have same years of experience
\end{itemize}

\end{frame}

\begin{frame}{Interpreting \(\hat{\beta_1}\) for continuous X}
\protect\hypertarget{interpreting-hatbeta_1-for-continuous-x}{}

\begin{itemize}
\tightlist
\item
  RQ: What is the effect of student teacher ratio (X1) on average
  district reading test scores (X2)?

  \begin{itemize}
  \tightlist
  \item
    \(Y_i = \beta_0 + \beta_1X_{1i} + \beta_2X_{2i} + \beta_3X_{3i} + u_i\)
  \item
    Where:

    \begin{itemize}
    \tightlist
    \item
      Y= reading test scores
    \item
      \(X_1\) = average district student teacher ratio, \(X_2\) = 0/1
      majority ELL district, \(X_3\) = avg district income (\$000s)
    \end{itemize}
  \item
    \textbf{IMPORTANT NOTE}: All independent variables should be at the
    same ``level''; here it is district!
  \end{itemize}
\item
  General interpretation of \(\hat{\beta_1}\) for continuous X

  \begin{itemize}
  \tightlist
  \item
    The avaerage effect of a one unit increase in \(X_1\) is a
    \(\hat{\beta_1}\) unit change in Y, holding the values of \(X_2\)
    and \(X_3\) constant
  \item
    OR The avaerage effect of a one unit increase in \(X_1\) is a
    \(\hat{\beta_1}\) unit change in Y, after controlling for \(X_2\)
    and \(X_3\)
  \item
    The avaerage effect of a one unit increase in \(X_1\) is a
    \(\hat{\beta_1}\) unit change in Y, holding the values of covariates
    constant
  \end{itemize}
\item
  Run regression in R!
\item
  \(\hat{Y_i} = \hat{\beta_0} + \hat{\beta_1} X_{1i} + \hat{\beta_2} X_{2i} + \hat{\beta_3} X_{3i}\)
\item
  \(\hat{Y_i} = 646.2 - 0.8 X_{1i} -23.5 X_{2i} + 1.7 X_{3i}\)
\item
  Specific example interpretation {[}run regression in R{]}

  \begin{itemize}
  \tightlist
  \item
    The average effect of a one-unit increase in average district
    student teacher ratio (i.e., one additional student per teacher) is
    a 0.8 point decrease in average district reading score, holding the
    values of majority ELL and district average income constant
  \end{itemize}
\end{itemize}

\end{frame}

\begin{frame}{Interpreting \(\hat{\beta_1}\) for categorical X}
\protect\hypertarget{interpreting-hatbeta_1-for-categorical-x}{}

\begin{itemize}
\tightlist
\item
  RQ: What is the effect of student teacher ratio (X1) on average
  district reading test scores (X2)?

  \begin{itemize}
  \tightlist
  \item
    \(Y_i = \beta_0 + \beta_1X_{1i} + \beta_2X_{2i} + \beta_3X_{3i} + u_i\)
  \item
    Where:

    \begin{itemize}
    \tightlist
    \item
      Y= reading test scores
    \item
      \(X_1\) = 0/1 majority ELL, \(X_2\) = avg district student teacher
      ratio \(X_3\) = avg district income (\$000s)
    \end{itemize}
  \item
    \textbf{Stylistic Note}: Your main independent variable of interest
    should always be \(X_1\)
  \end{itemize}
\item
  General interpretation of \(\hat{\beta_1}\) for categorical X

  \begin{itemize}
  \tightlist
  \item
    Being {[}non-reference group{]} as opposed to {[}reference group{]}
    is associated with a \(\hat{\beta_1}\) unit change in Y, holding the
    values of \(X_2\) and \(X_3\) constant
  \end{itemize}
\item
  Run regression in R {[}same coef values as previous model but in diff
  order{]}

  \begin{itemize}
  \tightlist
  \item
    \(\hat{Y_i} = \hat{\beta_0} + \hat{\beta_1} X_{1i} + \hat{\beta_2} X_{2i} + \hat{\beta_3} X_{3i}\)
  \item
    \(\hat{Y_i} = 646.2 - 23.5 X_{1i} -0.8 X_{2i} + 1.7 X_{3i}\)
  \end{itemize}
\item
  Specific interpretation

  \begin{itemize}
  \tightlist
  \item
    Reference group is the zero value of my dummy ELL var= non-ELL
    majority district; Non-Reference group is the one value of my dummy
    ELL var = majority ELL district
  \item
    Being a majority ELL district as opposed to a non-majority ELL
    district is associated with a 23 point decrease in average district
    reading scores, holding values of average student-teacher ratio and
    district average income constant
  \end{itemize}
\end{itemize}

\end{frame}

\begin{frame}{Prediction still works the same way!}
\protect\hypertarget{prediction-still-works-the-same-way}{}

\begin{itemize}
\tightlist
\item
  RQ: What is the effect of student teacher ratio (X1) on average
  district reading test scores (X2)?

  \begin{itemize}
  \tightlist
  \item
    \(Y_i = \beta_0 + \beta_1X_{1i} + \beta_2X_{2i} + \beta_3X_{3i} + u_i\)
  \item
    Where:

    \begin{itemize}
    \tightlist
    \item
      Y= reading test scores
    \item
      \(X_1\) = 0/1 majority ELL, \(X_2\) = avg district student teacher
      ratio \(X_3\) = avg district income (\$000s)
    \end{itemize}
  \end{itemize}
\item
  Run regression in R {[}same coef values as previous model but in diff
  order{]}

  \begin{itemize}
  \tightlist
  \item
    \(\hat{Y_i} = \hat{\beta_0} + \hat{\beta_1} X_{1i} + \hat{\beta_2} X_{2i} + \hat{\beta_3} X_{3i}\)
  \item
    \(\hat{Y_i} = 646.2 - 23.5 X_{1i} -0.8 X_{2i} + 1.7 X_{3i}\)
  \end{itemize}
\item
  What's the predicted average reading score for a district that is a
  non-ELL majority district, has a student teacher ratio of 25, and
  average district income of \$22,000?

  \begin{itemize}
  \tightlist
  \item
    \((Y| X_1=0, X_2=25, X_3=22)\) = 646.2 - (23.5 * 0) - (0.8 * 25) +
    (1.7 * 22)
  \item
    \((Y| X_1=0, X_2=25, X_3=22)\) = 646.2 - (0) - (20) + (37.4)
  \item
    \((Y| X_1=0, X_2=25, X_3=22)\) = 663.6
  \end{itemize}
\end{itemize}

\end{frame}

\begin{frame}{How to read regression results in academic journals}
\protect\hypertarget{how-to-read-regression-results-in-academic-journals}{}

\begin{itemize}
\tightlist
\item
  Cabrera et al (2014)

  \begin{itemize}
  \tightlist
  \item
    RQ: What is the effect of participating in MAS on high school
    graduation?
  \item
    Use program evaluation framework; but their model is a logistic
    regression because their Y= 0/1 graduated and X= 0/1 MAS
    participation
  \end{itemize}
\item
  Powers (2004)

  \begin{itemize}
  \tightlist
  \item
    RQ: what is relationship between school resource variables and
    school-level academic performance index (API)
  \item
    Don't frame article as ``program evaluation'' but it is! Y= School's
    academic performance index score X vars= school resource variables
  \end{itemize}
\item
  Regression results are pretty standardized across all fields and
  journals!

  \begin{itemize}
  \tightlist
  \item
    Regression tables usually show the coefficient and standard error
    (usually in parantheses) for each independent variable
  \item
    Columns are individual models!

    \begin{itemize}
    \tightlist
    \item
      Usually you start with a simple regression model that only
      includes your main independent variable of interest: ``model 1''
    \item
      Then you add controls; sometimes done in groupings
    \item
      Sometimes models in seperate columns also indicate various
      samples!
    \end{itemize}
  \end{itemize}
\end{itemize}

\end{frame}

\end{document}
