% Options for packages loaded elsewhere
\PassOptionsToPackage{unicode}{hyperref}
\PassOptionsToPackage{hyphens}{url}
\PassOptionsToPackage{dvipsnames,svgnames*,x11names*}{xcolor}
%
\documentclass[
  8pt,
  ignorenonframetext,
  dvipsnames]{beamer}
\usepackage{pgfpages}
\setbeamertemplate{caption}[numbered]
\setbeamertemplate{caption label separator}{: }
\setbeamercolor{caption name}{fg=normal text.fg}
\beamertemplatenavigationsymbolsempty
% Prevent slide breaks in the middle of a paragraph
\widowpenalties 1 10000
\raggedbottom
\setbeamertemplate{part page}{
  \centering
  \begin{beamercolorbox}[sep=16pt,center]{part title}
    \usebeamerfont{part title}\insertpart\par
  \end{beamercolorbox}
}
\setbeamertemplate{section page}{
  \centering
  \begin{beamercolorbox}[sep=12pt,center]{part title}
    \usebeamerfont{section title}\insertsection\par
  \end{beamercolorbox}
}
\setbeamertemplate{subsection page}{
  \centering
  \begin{beamercolorbox}[sep=8pt,center]{part title}
    \usebeamerfont{subsection title}\insertsubsection\par
  \end{beamercolorbox}
}
\AtBeginPart{
  \frame{\partpage}
}
\AtBeginSection{
  \ifbibliography
  \else
    \frame{\sectionpage}
  \fi
}
\AtBeginSubsection{
  \frame{\subsectionpage}
}
\usepackage{lmodern}
\usepackage{amssymb,amsmath}
\usepackage{ifxetex,ifluatex}
\ifnum 0\ifxetex 1\fi\ifluatex 1\fi=0 % if pdftex
  \usepackage[T1]{fontenc}
  \usepackage[utf8]{inputenc}
  \usepackage{textcomp} % provide euro and other symbols
\else % if luatex or xetex
  \usepackage{unicode-math}
  \defaultfontfeatures{Scale=MatchLowercase}
  \defaultfontfeatures[\rmfamily]{Ligatures=TeX,Scale=1}
\fi
% Use upquote if available, for straight quotes in verbatim environments
\IfFileExists{upquote.sty}{\usepackage{upquote}}{}
\IfFileExists{microtype.sty}{% use microtype if available
  \usepackage[]{microtype}
  \UseMicrotypeSet[protrusion]{basicmath} % disable protrusion for tt fonts
}{}
\makeatletter
\@ifundefined{KOMAClassName}{% if non-KOMA class
  \IfFileExists{parskip.sty}{%
    \usepackage{parskip}
  }{% else
    \setlength{\parindent}{0pt}
    \setlength{\parskip}{6pt plus 2pt minus 1pt}}
}{% if KOMA class
  \KOMAoptions{parskip=half}}
\makeatother
\usepackage{xcolor}
\IfFileExists{xurl.sty}{\usepackage{xurl}}{} % add URL line breaks if available
\IfFileExists{bookmark.sty}{\usepackage{bookmark}}{\usepackage{hyperref}}
\hypersetup{
  pdftitle={Introduction to Multivariate Regression \& Econometrics},
  pdfauthor={Lecture 7},
  colorlinks=true,
  linkcolor=Maroon,
  filecolor=Maroon,
  citecolor=Blue,
  urlcolor=blue,
  pdfcreator={LaTeX via pandoc}}
\urlstyle{same} % disable monospaced font for URLs
\newif\ifbibliography
\usepackage{color}
\usepackage{fancyvrb}
\newcommand{\VerbBar}{|}
\newcommand{\VERB}{\Verb[commandchars=\\\{\}]}
\DefineVerbatimEnvironment{Highlighting}{Verbatim}{commandchars=\\\{\}}
% Add ',fontsize=\small' for more characters per line
\usepackage{framed}
\definecolor{shadecolor}{RGB}{248,248,248}
\newenvironment{Shaded}{\begin{snugshade}}{\end{snugshade}}
\newcommand{\AlertTok}[1]{\textcolor[rgb]{0.94,0.16,0.16}{#1}}
\newcommand{\AnnotationTok}[1]{\textcolor[rgb]{0.56,0.35,0.01}{\textbf{\textit{#1}}}}
\newcommand{\AttributeTok}[1]{\textcolor[rgb]{0.77,0.63,0.00}{#1}}
\newcommand{\BaseNTok}[1]{\textcolor[rgb]{0.00,0.00,0.81}{#1}}
\newcommand{\BuiltInTok}[1]{#1}
\newcommand{\CharTok}[1]{\textcolor[rgb]{0.31,0.60,0.02}{#1}}
\newcommand{\CommentTok}[1]{\textcolor[rgb]{0.56,0.35,0.01}{\textit{#1}}}
\newcommand{\CommentVarTok}[1]{\textcolor[rgb]{0.56,0.35,0.01}{\textbf{\textit{#1}}}}
\newcommand{\ConstantTok}[1]{\textcolor[rgb]{0.00,0.00,0.00}{#1}}
\newcommand{\ControlFlowTok}[1]{\textcolor[rgb]{0.13,0.29,0.53}{\textbf{#1}}}
\newcommand{\DataTypeTok}[1]{\textcolor[rgb]{0.13,0.29,0.53}{#1}}
\newcommand{\DecValTok}[1]{\textcolor[rgb]{0.00,0.00,0.81}{#1}}
\newcommand{\DocumentationTok}[1]{\textcolor[rgb]{0.56,0.35,0.01}{\textbf{\textit{#1}}}}
\newcommand{\ErrorTok}[1]{\textcolor[rgb]{0.64,0.00,0.00}{\textbf{#1}}}
\newcommand{\ExtensionTok}[1]{#1}
\newcommand{\FloatTok}[1]{\textcolor[rgb]{0.00,0.00,0.81}{#1}}
\newcommand{\FunctionTok}[1]{\textcolor[rgb]{0.00,0.00,0.00}{#1}}
\newcommand{\ImportTok}[1]{#1}
\newcommand{\InformationTok}[1]{\textcolor[rgb]{0.56,0.35,0.01}{\textbf{\textit{#1}}}}
\newcommand{\KeywordTok}[1]{\textcolor[rgb]{0.13,0.29,0.53}{\textbf{#1}}}
\newcommand{\NormalTok}[1]{#1}
\newcommand{\OperatorTok}[1]{\textcolor[rgb]{0.81,0.36,0.00}{\textbf{#1}}}
\newcommand{\OtherTok}[1]{\textcolor[rgb]{0.56,0.35,0.01}{#1}}
\newcommand{\PreprocessorTok}[1]{\textcolor[rgb]{0.56,0.35,0.01}{\textit{#1}}}
\newcommand{\RegionMarkerTok}[1]{#1}
\newcommand{\SpecialCharTok}[1]{\textcolor[rgb]{0.00,0.00,0.00}{#1}}
\newcommand{\SpecialStringTok}[1]{\textcolor[rgb]{0.31,0.60,0.02}{#1}}
\newcommand{\StringTok}[1]{\textcolor[rgb]{0.31,0.60,0.02}{#1}}
\newcommand{\VariableTok}[1]{\textcolor[rgb]{0.00,0.00,0.00}{#1}}
\newcommand{\VerbatimStringTok}[1]{\textcolor[rgb]{0.31,0.60,0.02}{#1}}
\newcommand{\WarningTok}[1]{\textcolor[rgb]{0.56,0.35,0.01}{\textbf{\textit{#1}}}}
\usepackage{longtable,booktabs}
\usepackage{caption}
% Make caption package work with longtable
\makeatletter
\def\fnum@table{\tablename~\thetable}
\makeatother
\setlength{\emergencystretch}{3em} % prevent overfull lines
\providecommand{\tightlist}{%
  \setlength{\itemsep}{0pt}\setlength{\parskip}{0pt}}
\setcounter{secnumdepth}{-\maxdimen} % remove section numbering

%packages
\usepackage{graphicx}
\usepackage{rotating}
\usepackage{hyperref}

\usepackage{tikz} % used for text highlighting, amongst others
\usepackage{comment}

%title slide stuff
%\institute{Department of Education}
%\title{Managing and Manipulating Data Using R}

%
\setbeamertemplate{navigation symbols}{} % get rid of navigation icons:
\setbeamertemplate{footline}[page number]

%\setbeamertemplate{frametitle}{\thesection \hspace{0.2cm} \insertframetitle}
\setbeamertemplate{section in toc}[sections numbered]
%\setbeamertemplate{subsection in toc}[subsections numbered]
\setbeamertemplate{subsection in toc}{%
  \leavevmode\leftskip=3.2em\color{gray}\rlap{\hskip-2em\inserttocsectionnumber.\inserttocsubsectionnumber}\inserttocsubsection\par
}

%define colors
%\definecolor{uva_orange}{RGB}{216,141,42} % UVa orange (Rotunda orange)
\definecolor{mygray}{rgb}{0.95, 0.95, 0.95} % for highlighted text
	% grey is equal parts red, green, blue. higher values >> lighter grey
	%\definecolor{lightgraybo}{rgb}{0.83, 0.83, 0.83}

% new commands

%highlight text with very light grey
\newcommand*{\hlg}[1]{%
	\tikz[baseline=(X.base)] \node[rectangle, fill=mygray] (X) {#1};%
}
%, inner sep=0.3mm
%highlight text with very light grey and use font associated with code
\newcommand*{\hlgc}[1]{\texttt{\hlg{#1}}}

%modifying back ticks to add grey background
\let\OldTexttt\texttt
\renewcommand{\texttt}[1]{\OldTexttt{\hlg{#1}}}


\begin{comment}

% Font
\usepackage[defaultfam,light,tabular,lining]{montserrat}
\usepackage[T1]{fontenc}
\renewcommand*\oldstylenums[1]{{\fontfamily{Montserrat-TOsF}\selectfont #1}}

% Change color of boldface text to darkgray
\renewcommand{\textbf}[1]{{\color{darkgray}\bfseries\fontfamily{Montserrat-TOsF}#1}}

% Bullet points
\setbeamertemplate{itemize item}{\color{BlueViolet}$\circ$}
\setbeamertemplate{itemize subitem}{\color{BrickRed}$\triangleright$}
\setbeamertemplate{itemize subsubitem}{$-$}

% Reduce space before lists
%\addtobeamertemplate{itemize/enumerate body begin}{}{\vspace*{-8pt}}

\let\olditem\item
\renewcommand{\item}{%
  \olditem\vspace{4pt}
}

% decreasing space before and after level-2 bullet block
%\addtobeamertemplate{itemize/enumerate subbody begin}{}{\vspace*{-3pt}}
%\addtobeamertemplate{itemize/enumerate subbody end}{}{\vspace*{-3pt}}

% decreasing space before and after level-3 bullet block
%\addtobeamertemplate{itemize/enumerate subsubbody begin}{}{\vspace*{-2pt}}
%\addtobeamertemplate{itemize/enumerate subsubbody end}{}{\vspace*{-2pt}}

%Section numbering
\setbeamertemplate{section page}{%
    \begingroup
        \begin{beamercolorbox}[sep=10pt,center,rounded=true,shadow=true]{section title}
        \usebeamerfont{section title}\thesection~\insertsection\par
        \end{beamercolorbox}
    \endgroup
}

\setbeamertemplate{subsection page}{%
    \begingroup
        \begin{beamercolorbox}[sep=6pt,center,rounded=true,shadow=true]{subsection title}
        \usebeamerfont{subsection title}\thesection.\thesubsection~\insertsubsection\par
        \end{beamercolorbox}
    \endgroup
}

\end{comment}

\title{Introduction to Multivariate Regression \& Econometrics}
\subtitle{HED 612}
\author{Lecture 7}
\date{}

\begin{document}
\frame{\titlepage}

\begin{frame}
  \tableofcontents[hideallsubsections]
\end{frame}
\begin{frame}{Download Data and Open R Script}
\protect\hypertarget{download-data-and-open-r-script}{}

We'll be using GSS and CA Data!

\medskip

\begin{enumerate}
\tightlist
\item
  Download the Lecture 7 PDF and R files for this week

  \begin{itemize}
  \tightlist
  \item
    Place all files in HED612\_S21
    \textgreater\textgreater\textgreater{} lectures
    \textgreater\textgreater\textgreater{} lecture7
  \end{itemize}
\item
  Open the RProject (should be in your main HED612\_S21 folder)
\item
  Once the RStudio window opens, open the Lecture 7 R script by clicking
  on:

  \begin{itemize}
  \tightlist
  \item
    file \textgreater\textgreater\textgreater{} open file\ldots{}
    \textgreater\textgreater\textgreater{} {[}navigate to lecture 7
    folder{]} \textgreater\textgreater\textgreater{} lecture7.R
  \end{itemize}
\end{enumerate}

\end{frame}

\begin{frame}{Schedule}
\protect\hypertarget{schedule}{}

\textbf{What we have done:}

\begin{itemize}
\tightlist
\item
  Prediction
\item
  Population Regression Model \& OLS Prediction Line {[}will review
  today{]}
\item
  Interpretation of \(\hat{\beta_1}\) with continuous X {[}will review
  today{]}
\end{itemize}

\medskip

\textbf{Today:}

\begin{itemize}
\tightlist
\item
  Confidence Intervals for \(\hat{\beta_0}\) and \(\hat{\beta_1}\)
\item
  Mini R lesson on creating new variables
\item
  Interpretation of \(\hat{\beta_1}\) with categorical X

  \begin{itemize}
  \tightlist
  \item
    Homework 7 posted on D2L
  \item
    Reading for next week: TBD {[}on Omitted Variable Bias{]}
  \end{itemize}
\end{itemize}

\medskip

\textbf{Next week {[}3/3/2021{]}:}

\begin{itemize}
\tightlist
\item
  Review SER vs SE of \(\hat{\beta_1}\)
\item
  OLS Assumptions
\item
  Introduction to Omitted Variable Bias
\item
  Final project requirements and intro to possible datasets!
\end{itemize}

\medskip

\textbf{3/10/2021: Spring Break/Reading Day {[}No Class{]}}

\textbf{3/17/2021: Intro to Multivariate Regression}

\end{frame}

\begin{frame}{General Regression Purposes}
\protect\hypertarget{general-regression-purposes}{}

Things we generally do with regression:

\begin{itemize}
\tightlist
\item
  Prediction

  \begin{itemize}
  \tightlist
  \item
    Here we're interested in knowing/predicting \(\hat{Y}\)
  \item
    Example: Predict poverty status from owning a cell phone
  \item
    Example: Predict academic probation for early warning system

    \begin{itemize}
    \tightlist
    \item
      We don't really care which X variables predict academic
      probation\ldots{} (i.e., absences, going to REC center 2+ times a
      week, etc.)
    \end{itemize}
  \end{itemize}
\end{itemize}

\medskip

\begin{itemize}
\tightlist
\item
  Hypothesis Testing about \(\beta_1\)

  \begin{itemize}
  \tightlist
  \item
    Here we're interested primarily the impact our X has on Y, in other
    words the slope and significance of \(\hat{\beta_1}\)
  \item
    Example: Does smaller class sizes cause better student learning
  \item
    Example: Does receiving federal financial aid have an effect on
    on-time graduation
  \item
    Our focus is on our one independent variable of interest, and
    sometimes test the effect of X on multiple Y's (i.e., on-time
    graduation, first to second year retention, GPA, etc)
  \end{itemize}
\end{itemize}

\end{frame}

\begin{frame}{Population Regression Model and OLS Prediction Line}
\protect\hypertarget{population-regression-model-and-ols-prediction-line}{}

RQ: What is the effect of student-teacher ratio (X) on student test
scores (Y)

\textbf{Population Linear Regression Model}:
\(Y_i = \beta_0 + \beta_1X_i + u_i\)

\begin{itemize}
\tightlist
\item
  Where Y = student test scores
\item
  Where X = student teacher ratio
\end{itemize}

\medskip

\textbf{OLS Prediction Line or ``OLS Regression Line'' (without
estimates)}: \(\hat{Y_i} = \hat{\beta_0} + \hat{\beta_1}X_i\)

\begin{itemize}
\tightlist
\item
  We DROP the residual term! Why?
\item
  Residuals = everything not included in the model that account for the
  difference between actual observed value of Y and Y value predicted by
  OLS regression
\item
  We can only predict values of Y based on data we have!
\item
  We use residuals to understand how good our predictions are! (SER)
\end{itemize}

\end{frame}

\begin{frame}[fragile]{Interpretation of \(\hat{\beta_1}\) for Continous
X}
\protect\hypertarget{interpretation-of-hatbeta_1-for-continous-x}{}

\textbf{OLS Prediction Line or ``OLS Regression Line'' (with
estimates)}: \(\hat{Y_i} = \hat{\beta_0} + \hat{\beta_1}X_i\)

\medskip

\textbf{Interpretation of \(\hat{\beta_1}\)}

\begin{itemize}
\tightlist
\item
  General interpretation {[}always true!{]}

  \begin{itemize}
  \tightlist
  \item
    The average effect of a one-unit increase in X is associated with a
    \(\hat{\beta_1}\) unit change (negative = decrease or positive =
    increase) in Y
  \end{itemize}
\end{itemize}

\begin{Shaded}
\begin{Highlighting}[]
  \KeywordTok{summary}\NormalTok{(mod1)}
\CommentTok{#> }
\CommentTok{#> Call:}
\CommentTok{#> lm(formula = testscr ~ str, data = caschool)}
\CommentTok{#> }
\CommentTok{#> Residuals:}
\CommentTok{#>     Min      1Q  Median      3Q     Max }
\CommentTok{#> -47.727 -14.251   0.483  12.822  48.540 }
\CommentTok{#> }
\CommentTok{#> Coefficients:}
\CommentTok{#>             Estimate Std. Error t value Pr(>|t|)    }
\CommentTok{#> (Intercept) 698.9330     9.4675  73.825  < 2e-16 ***}
\CommentTok{#> str          -2.2798     0.4798  -4.751 2.78e-06 ***}
\CommentTok{#> ---}
\CommentTok{#> Signif. codes:  0 '***' 0.001 '**' 0.01 '*' 0.05 '.' 0.1 ' ' 1}
\CommentTok{#> }
\CommentTok{#> Residual standard error: 18.58 on 418 degrees of freedom}
\CommentTok{#> Multiple R-squared:  0.05124,    Adjusted R-squared:  0.04897 }
\CommentTok{#> F-statistic: 22.58 on 1 and 418 DF,  p-value: 2.783e-06}
\end{Highlighting}
\end{Shaded}

\end{frame}

\begin{frame}{Point and Interval Estimation}
\protect\hypertarget{point-and-interval-estimation}{}

\begin{itemize}
\tightlist
\item
  Paramater

  \begin{itemize}
  \tightlist
  \item
    A summary of the population; usually unknown
  \end{itemize}
\end{itemize}

\medskip

\begin{itemize}
\tightlist
\item
  Point Estimate

  \begin{itemize}
  \tightlist
  \item
    A single number that is the best guess for the parameter (e.g.,
    \(\hat{\beta_1}\))
  \item
    Use hypothesis testing to explore whether there is a significant
    relationship between X and Y

    \begin{itemize}
    \tightlist
    \item
      \(H_0: \beta_1 = 0\) (no relationship: no effect of X on Y)
    \item
      \(H_0: \beta_1 \ne 0\) (there is a relationship: X does have an
      effect on Y)
    \end{itemize}
  \end{itemize}
\end{itemize}

\medskip

\begin{itemize}
\tightlist
\item
  Interval Estimate

  \begin{itemize}
  \tightlist
  \item
    An interval around the point estimate, within which the parameter
    value is believed to fall
  \item
    e.g., If \(\hat{\beta_1}\)= 2.5, we are 95\% sure that
    \(\hat{\beta_1}\) is between 1.5 and 3.5
  \end{itemize}
\end{itemize}

\end{frame}

\hypertarget{confidence-intervals}{%
\section{Confidence Intervals}\label{confidence-intervals}}

\begin{frame}{Confidence intervals about \(\beta_1\)}
\protect\hypertarget{confidence-intervals-about-beta_1}{}

\begin{itemize}
\tightlist
\item
  General formula for confidence intervals

  \begin{itemize}
  \tightlist
  \item
    (point estimate) \(\pm\) z*SE(point estimate)
  \item
    Where z = z-score associated with desired confidence interval
  \end{itemize}
\end{itemize}

\begin{longtable}[]{@{}ll@{}}
\toprule
Confidence Interval & Z-Score\tabularnewline
\midrule
\endhead
90\% & 1.645\tabularnewline
95\% & 1.96\tabularnewline
99\% & 2.576\tabularnewline
\bottomrule
\end{longtable}

\medskip

\begin{itemize}
\tightlist
\item
  Formulas for 95\% confidence interval (CI) of \(\beta_1\)

  \begin{itemize}
  \tightlist
  \item
    \(\hat{\beta_1}\) \(\pm\) 1.96 * SE(\(\hat{\beta_1}\))
  \item
    Interpretation: We are 95\% confident that the population parameter
    \(\beta_1\) lies somewhere between {[}lower bound{]} and {[}upper
    bound{]}
  \end{itemize}
\end{itemize}

\medskip

\begin{itemize}
\tightlist
\item
  What happens to CI when you choose a higher ``confidence interval''
  level? Why?

  \begin{itemize}
  \tightlist
  \item
    e.g., 99\% CI instead of 95\%
  \end{itemize}
\end{itemize}

\end{frame}

\begin{frame}[fragile]{Confidence Interval in R}
\protect\hypertarget{confidence-interval-in-r}{}

RQ: What is the effect of district average income (in \$000s) (X) on
student test scores (Y)?

\begin{itemize}
\tightlist
\item
  Write out population regression model
\item
  Write out OLS regression without estimates
\end{itemize}

\medskip

Run regression in R

\begin{itemize}
\tightlist
\item
  Calculate 95\% CI using \(\hat{\beta_1}\) and SE(\(\hat{\beta_1}\))

  \begin{itemize}
  \tightlist
  \item
    We are 95\% confident that the population parameter \(\beta_1\) lies
    somewhere between 1.70 and 2.06
  \end{itemize}
\item
  Calculate 99\% CI using \(\hat{\beta_1}\) and SE(\(\hat{\beta_1}\))

  \begin{itemize}
  \tightlist
  \item
    We are 99\% confident that the population parameter \(\beta_1\) lies
    somewhere between 1.64 and 2.11
  \end{itemize}
\end{itemize}

\medskip

Use \texttt{confint()} in R to calculate CI

\end{frame}

\begin{frame}{Confidence Intervals and Hypothesis Testing}
\protect\hypertarget{confidence-intervals-and-hypothesis-testing}{}

We always test same null hypothesis

\begin{itemize}
\tightlist
\item
  \(H_0: \beta_1 = 0\)
\item
  Reject \(H_0\) if p-value is less than ``alpha-level''
\end{itemize}

\medskip

Relationship between confidence intervals and hypothesis tests about
\(\beta_1\)

\begin{itemize}
\tightlist
\item
  Assume testing \(H_0\) with alpha-level = .05

  \begin{itemize}
  \tightlist
  \item
    If p-value for \(H_0\) is less than .05, then 95\% CI will not
    contain zero (our value associated with the null)
  \item
    If 95\% CI does not contain zero (our value associated with the
    null), then p-value for \(H_0\) is less than .05
  \end{itemize}
\item
  Assume testing \(H_0\) with alpha-level = .01

  \begin{itemize}
  \tightlist
  \item
    If p-value for \(H_0\) is less than .01, then 99\% CI will not
    contain zero (our value associated with the null)
  \item
    If 99\% CI does not contain zero (our value associated with the
    null), then p-value for \(H_0\) is less than .01
  \end{itemize}
\end{itemize}

\end{frame}

\begin{frame}[fragile]{Student Exercise \#1}
\protect\hypertarget{student-exercise-1}{}

Using CA Schools data: RQ: What is the effect of student teacher ratio
(X) on test scores (Y)?

\begin{enumerate}
\item
  Write out population regression model for the effect of student
  teacher ratio on district test scores?
\item
  Run the regression in R as \texttt{stuex\_mod}. Write the OLS
  prediction line with estimates
\item
  What is the point estimate for \(\beta_1\)? Interpret this estimate.
\item
  Using R's \texttt{confint()} function, what is the 95\% CI for
  \(\beta_1\)? Interpret in words.
\item
  Calculate the 99\% on your own using \(\hat{\beta_1}\) and
  SE(\(\hat{\beta_1}\)).
\end{enumerate}

\end{frame}

\begin{frame}[fragile]{Student Exercise \#1 {[}Solutions!{]}}
\protect\hypertarget{student-exercise-1-solutions}{}

\begin{enumerate}
\tightlist
\item
  Write out population regression model for the effect of student
  teacher ratio on district test scores?
\end{enumerate}

\begin{itemize}
\tightlist
\item
  \(Y_i = \beta_0 + \beta_1X_i + u_i\)

  \begin{itemize}
  \tightlist
  \item
    Where Y = \texttt{testscr}
  \item
    Where X = \texttt{str}
  \end{itemize}
\end{itemize}

\begin{enumerate}
\setcounter{enumi}{1}
\tightlist
\item
  Run the regression in R as \texttt{stuex\_mod}. Write the OLS
  prediction line with estimates
\end{enumerate}

\begin{itemize}
\tightlist
\item
  \(\hat{Y_i} = \hat{\beta_0} + \hat{\beta_1}X_i\)
\item
  \(\hat{Y_i} = 699 + (-2.28)X_i\)
\end{itemize}

\begin{enumerate}
\setcounter{enumi}{2}
\tightlist
\item
  What is the point estimate for \(\beta_1\)? Interpret this estimate.
\end{enumerate}

\begin{itemize}
\tightlist
\item
  Point estimate for \(\beta_1\): \(\hat{\beta_1}\) = 2.28; A one unit
  increase in student teacher ratio is associated with a 2.28 point
  decrease, on average, on district student test scores
\end{itemize}

\begin{enumerate}
\setcounter{enumi}{3}
\tightlist
\item
  Using R's \texttt{confint()} function, what is the 95\% CI for
  \(\beta_1\)? Interpret in words.
\end{enumerate}

\begin{itemize}
\tightlist
\item
  \texttt{confint(stuex\_mod,\ level\ =\ 0.95)}: We are 95\% confident
  that the population parameter \(\beta_1\) lies somewhere between -3.22
  and -1.34
\end{itemize}

\begin{enumerate}
\setcounter{enumi}{4}
\tightlist
\item
  Calculate the 99\% on your own using \(\hat{\beta_1}\) and
  SE(\(\hat{\beta_1}\)). - \(\hat{\beta_1}\) \(\pm\) 2.576 *
  SE(\(\hat{\beta_1}\))
\end{enumerate}

\begin{itemize}
\tightlist
\item
  -2.28 \(\pm\) 2.576 * SE(0.4798)
\item
  -2.28 \(\pm\) 1.235965
\item
  -3.52, -1.04
\end{itemize}

\end{frame}

\hypertarget{creating-variables-in-r}{%
\section{Creating variables in R}\label{creating-variables-in-r}}

\begin{frame}{Creating ``analysis'' variables in R}
\protect\hypertarget{creating-analysis-variables-in-r}{}

\begin{itemize}
\tightlist
\item
  Quantitative researchers really need two different skills

  \begin{itemize}
  \tightlist
  \item
    \textbf{Statistics}: learning how to run and interpret apppropriate
    statistical tests/methods according to RQs; learning how to apply
    findings to answer RQs; draw recommendations and implications from
    statistical findings
  \item
    \textbf{Data Management}: proficiency in fundamental data management
    and manipulation tasks like managing data in their raw forms,
    creating variables, combining multiple datasets, manipulation or
    reshaping data, etc.
  \end{itemize}
\item
  This is not a ``data management'' class; or a class to teach you how
  to use R
\item
  We simply use R to run the statistical tests associated with linear
  regression
\item
  We, for the most part, use data that is ``clean''; however you often
  need to create different ``analysis'' versions of variables
\item
  We are gonna have a ``crash course'' lecture on creating categorical
  variables today.

  \begin{itemize}
  \tightlist
  \item
    We'll start on learning this via creating a dummy variable; then in
    following weeks we will create variables with 2+ categories
  \end{itemize}
\end{itemize}

\end{frame}

\begin{frame}[fragile]{``Coneptual'' Process for Creating ``analysis''
variables in R}
\protect\hypertarget{coneptual-process-for-creating-analysis-variables-in-r}{}

\begin{enumerate}
\tightlist
\item
  Investigating values and patterns of variables from ``input data''
\item
  Identifying and cleaning errors or values that need to be changed
\item
  Creating ``analysis'' variables
\item
  Checking values of analysis variables against values of input
  variables
\end{enumerate}

\medskip

\begin{itemize}
\tightlist
\item
  Example: Create a new dummy variable (\texttt{ba\_degree}) for whether
  a respondent has at least a Bachelor's degree (\texttt{ba\_degree}= 1)
  or less than a Bachelor's degree (\texttt{ba\_degree}= 0)
\end{itemize}

\end{frame}

\begin{frame}[fragile]{Creating variables in R}
\protect\hypertarget{creating-variables-in-r-1}{}

Task: Create a new dummy version of the variable \texttt{degree} called
\texttt{ba\_degree}

\begin{itemize}
\tightlist
\item
  1 indicates respondent has at least a Bachelor's degree
\item
  0 indicates respondent has less than a Bachelor's degree
\end{itemize}

\medskip

\textbf{Step 1: Investigate values and patterns of variable from ``input
data''}

\begin{itemize}
\tightlist
\item
  use \texttt{var\_label()} and \texttt{val\_labels()} to check variable
  and value labels
\item
  use \texttt{count()} to get a frequency count of each category
\item
  use \texttt{count()} + \texttt{is.na()} to check if there are any
  missing observations
\end{itemize}

\medskip

\textbf{Step 2: Identifying and cleaning errors or values that need to
be changed}

\begin{itemize}
\tightlist
\item
  Most common cleaning error to fix: National surveys don't report
  missing as \texttt{NA}; they assign ``strange'' values to missing
  observations

  \begin{itemize}
  \tightlist
  \item
    \texttt{-99}= \texttt{item\ legitmate\ skip}
  \item
    \texttt{-98} = \texttt{no\ response}
  \end{itemize}
\item
  Use \texttt{mutate()} to fix errors

  \begin{itemize}
  \tightlist
  \item
    Create version of ``input'' variables that code missing values
    (-99,-98) as true missing in R (\texttt{NA}).
  \item
    Our \texttt{degree} variable is actually clean already! All missing
    categories are already coded to \texttt{NA} (they were previously
    IAP=-8, DK=-9, and NA=-7 )
  \item
    We'll practice this step next week on a variable that is not so
    clean!
  \end{itemize}
\end{itemize}

\end{frame}

\begin{frame}[fragile]{Creating variables in R}
\protect\hypertarget{creating-variables-in-r-2}{}

Task: Create a new dummy variable (\texttt{ba\_degree})

\begin{itemize}
\tightlist
\item
  1 indicates respondent has at least a Bachelor's degree
\item
  0 indicates respondent has less than a Bachelor's degree
\end{itemize}

\medskip

\textbf{Step 3: Creating ``analysis'' variables}

\begin{itemize}
\tightlist
\item
  Create dummy vars via \texttt{mutate()} + \texttt{ifelse()}; where
  general syntax is:
\end{itemize}

\begin{Shaded}
\begin{Highlighting}[]
\NormalTok{ df <-}\StringTok{ }\NormalTok{df }\OperatorTok\StringTok{ }
\StringTok{  }\KeywordTok{mutate}\NormalTok{(}\DataTypeTok{NEWVAR=} 
           \KeywordTok{ifelse}\NormalTok{(OLDVAR}\OperatorTok{+}\NormalTok{CONDITION, value }\ControlFlowTok{if} \OtherTok{TRUE}\NormalTok{, value }\ControlFlowTok{if} \OtherTok{FALSE}\NormalTok{))}
\end{Highlighting}
\end{Shaded}

\medskip

\textbf{Step 4: Checking values of analysis variables against values of
input variables}

\begin{itemize}
\tightlist
\item
  Use \texttt{group\_by()\ +\ count()} to check new and old variables
  against each other
\end{itemize}

\medskip

Warning: You will ONLY use the assignment operator \texttt{\textless{}-}
within Steps 2 and Steps 3; using \texttt{\textless{}-} in Step 1 or
Step 4 will change the original \texttt{GSS}dataset

\begin{itemize}
\tightlist
\item
  If you make this mistake; just reload your gss dataset!
\item
  show in R
\end{itemize}

\end{frame}

\hypertarget{minute-break}{%
\section{10 Minute Break}\label{minute-break}}

\hypertarget{interpretation-of-hatbeta_1-with-categorical-x}{%
\section{\texorpdfstring{Interpretation of \(\hat{\beta_1}\) with
Categorical
X}{Interpretation of \textbackslash hat\{\textbackslash beta\_1\} with Categorical X}}\label{interpretation-of-hatbeta_1-with-categorical-x}}

\begin{frame}{Interpretation of \(\hat{\beta_1}\) with Categorical X}
\protect\hypertarget{interpretation-of-hatbeta_1-with-categorical-x-1}{}

\begin{itemize}
\item
  Many independent variables of interest are categorical rather than
  continous
\item
  How to distinguish between continuous and categorical variables when
  running regression?

  \begin{itemize}
  \tightlist
  \item
    Continous variables:

    \begin{itemize}
    \tightlist
    \item
      Difference between one value and another is quantitative
    \item
      e.g., SAT score of 900 vs.~1000; income of \$40k vs \$45k, GPA of
      2.0 vs.~2.1
    \end{itemize}
  \item
    Categorical variables:

    \begin{itemize}
    \tightlist
    \item
      Difference between one value and another cannot be measured
      quantitatively
    \item
      e.g., race/ethnicity, parent education is B.A. vs M.D., political
      ideology
    \end{itemize}
  \end{itemize}
\end{itemize}

\medskip

Many program evaluation questions involve a categorical independent
variable of interest

\begin{itemize}
\tightlist
\item
  What is the effect of receiving a pell grant on on-time college
  completion?
\item
  What is the effect of participating in Mexican American Studies
  program on high school graduation?
\item
  What is the effect of Head Start pre-k on Kindergarten reading levels?
\item
  What is the effect of class size (small vs large) on student learning?
\end{itemize}

\end{frame}

\begin{frame}{General Steps for Regression with Categorical X}
\protect\hypertarget{general-steps-for-regression-with-categorical-x}{}

\begin{itemize}
\tightlist
\item
  Identify categories of X and choose a reference group
\item
  Create 0/1 variables for each group
\item
  Write out population model and OLS regression line
\item
  Run regression in R
\item
  Interpret estimates
\end{itemize}

\end{frame}

\begin{frame}[fragile]{Interpretation of \(\hat{\beta_1}\) with
Categorical X}
\protect\hypertarget{interpretation-of-hatbeta_1-with-categorical-x-2}{}

\begin{itemize}
\tightlist
\item
  Y = income and X = college graduate (BA or higher)
\item
  Choose the ``reference group'' or ``base level''; this is who all
  other groups will be compared to (similar to ANOVA)

  \begin{itemize}
  \tightlist
  \item
    Non-college graduates will be our reference group
  \item
    College graduates will be our non-reference group
  \item
    If your categorical variable is a dummy variable, your reference
    category should be equal to zero and non-reference equal to 1!
  \item
    We already did this! \texttt{ba\_degree} = 0 for lower than a BA and
    1 = for BA or higher
  \end{itemize}
\end{itemize}

\medskip

\begin{itemize}
\tightlist
\item
  Population regression model

  \begin{itemize}
  \tightlist
  \item
    \(Y_i = \beta_0 + \beta_1X_i + u_i\)
  \item
    Where Y= income
  \item
    X= 0/1 college graduate

    \begin{itemize}
    \tightlist
    \item
      0 = non-college graduate {[}reference group{]}
    \item
      1 = college graduate {[}non-reference group{]}
    \end{itemize}
  \end{itemize}
\item
  OLS Prediction Line {[}run regression in R{]}

  \begin{itemize}
  \tightlist
  \item
    R will automatically assign the lowest value of X as your reference
    category!
  \item
    \(\hat{Y_i} = \hat{\beta_0} + \hat{\beta_1}X_i\)
  \item
    \(\hat{Y_i} = 17551.5 + 21090.2*X_i\)
  \end{itemize}
\item
  \textbf{Generic interpretation of \(\hat{\beta_1}\) for categorical X}

  \begin{itemize}
  \tightlist
  \item
    \textbf{The average effect of being {[}specific non-reference
    category{]} as opposed to {[}reference category{]} is associated
    with a \(\hat{\beta_1}\) change in Y}
  \end{itemize}
\item
  Specific interpretation of \(\hat{\beta_1}=21090.2\)

  \begin{itemize}
  \tightlist
  \item
    the average effect of having a \textbf{BA or higher} as opposed to
    having \textbf{less than a BA}, on average, is associated with a
    \$21,090.20 increase in annual income
  \end{itemize}
\end{itemize}

\end{frame}

\begin{frame}{Same Interpretation of \(\hat{\beta_1}\)}
\protect\hypertarget{same-interpretation-of-hatbeta_1}{}

\begin{itemize}
\tightlist
\item
  Generic interpretation of \(\hat{\beta_1}\) when X=continuous

  \begin{itemize}
  \tightlist
  \item
    The average effect of a one-unit increase in X is a
    \(\hat{\beta_1}\) unit change in the value of Y
  \end{itemize}
\item
  Generic interpretation of \(\hat{\beta_1}\) when X=categorical

  \begin{itemize}
  \tightlist
  \item
    X; 0= reference group; 1= non-reference group
  \item
    The average effect of being {[}non-reference group{]} as opposed to
    {[}reference group{]} is associated with a \(\hat{\beta_1}\) change
    in Y

    \begin{itemize}
    \tightlist
    \item
      In other words: a one-unit increase in X (from X=0 to X=1) is
      associated with \(\hat{\beta_1}\) change in Y
    \end{itemize}
  \end{itemize}
\end{itemize}

\medskip

So interpretation of \(\hat{\beta_1}\) for categorical X is the same as
for continous X

\begin{itemize}
\tightlist
\item
  In both cases \(\hat{\beta_1}\) is the effect of a one unit increase
  in X
\item
  But in categorical, X can only increase one unit (from X=0 to X=1)
\end{itemize}

\end{frame}

\begin{frame}[fragile]{Interpretation of \(\hat{\beta_1}\) with
Categorical X}
\protect\hypertarget{interpretation-of-hatbeta_1-with-categorical-x-3}{}

\begin{itemize}
\tightlist
\item
  What if we switch our reference category so that X is now a dummy
  variable where:

  \begin{itemize}
  \tightlist
  \item
    X=0 are BA or Higher
  \item
    X= 1 are lower than a BA
  \item
    make this variable \texttt{lower\_ba} in R
  \end{itemize}
\item
  Population regression model

  \begin{itemize}
  \tightlist
  \item
    \(Y_i = \beta_0 + \beta_1X_i + u_i\)
  \item
    Where Y= income, X= 0/1 non-college graduate
  \end{itemize}
\item
  OLS Prediction Line {[}run regression in R{]}

  \begin{itemize}
  \tightlist
  \item
    \(\hat{Y_i} = \hat{\beta_0} + \hat{\beta_1}X_i\)
  \item
    \(\hat{Y_i} = 38,642.5 + (-21090.2)*X_i\)
  \end{itemize}
\item
  Generic interpretation of \(\hat{\beta_1}\) for categorical X

  \begin{itemize}
  \tightlist
  \item
    the average effect of being {[}specific non-reference category{]} as
    opposed to {[}reference category{]} is associated with a
    \(\hat{\beta_1}\) change in Y
  \end{itemize}
\item
  Specific interpretation of \(\hat{\beta_1}=-21090\)

  \begin{itemize}
  \tightlist
  \item
    students answer
  \end{itemize}
\item
  How do we explain the differences in \(\hat{\beta_0}\) between these
  two regressions?

  \begin{itemize}
  \tightlist
  \item
    Model 2 {[}X=0 lower than BA, X=1 BA or higher{]}: \(\hat{\beta_0}\)
    = 17551.5
  \item
    Model 3 {[}X=0 BA or higher, X=1 lower than BA{]}: \(\hat{\beta_0}\)
    = 38642
  \end{itemize}
\end{itemize}

\end{frame}

\begin{frame}{Interpretation of \(\hat{\beta_1}\) with Categorical X
(more than 2 categories!)}
\protect\hypertarget{interpretation-of-hatbeta_1-with-categorical-x-more-than-2-categories}{}

\begin{itemize}
\tightlist
\item
  What is the effect of respondent political party on income?

  \begin{itemize}
  \tightlist
  \item
    Categories: democrat, republican, independent, unknown party
  \item
    We need to create dummy variables for each of these categories
    first! {[}show in R{]}
  \end{itemize}
\end{itemize}

\medskip

\begin{itemize}
\tightlist
\item
  Choose category that will be our ``reference group''

  \begin{itemize}
  \tightlist
  \item
    Let's choose democrats as the reference group!
  \item
    When we have more than one category; we run the regression with all
    category dummies \emph{except} the reference group!
  \end{itemize}
\end{itemize}

\medskip

\begin{itemize}
\tightlist
\item
  Population regression model

  \begin{itemize}
  \tightlist
  \item
    \(Y_i = \beta_0 + \beta_1X_{1i} + \beta_2X_{2i} + \beta_3X_{3i} + u_i\)
  \item
    Where Y= income;
  \item
    \(X_{1}\)= 0/1 republican, \(X_{2}\) = 0/1 independent, \(X_{3}\) =
    0/1 unknown party; Reference Category = democrats
  \end{itemize}
\end{itemize}

\medskip

\begin{itemize}
\tightlist
\item
  OLS Prediction Line {[}run regression in R{]}

  \begin{itemize}
  \tightlist
  \item
    \(\hat{Y_i} = \hat{\beta_0} + \hat{\beta_1}X_{1i} + \hat{\beta_2}X_{2i} + \hat{\beta_3}X_{3i}\)
  \item
    \(\hat{Y_i} = 24476 + 6184*X_{1i} + (-1736)*X_{2i} + (-3335)*X_{3i}\)
  \end{itemize}
\end{itemize}

\end{frame}

\begin{frame}{Interpretation of \(\hat{\beta_1}\) with Categorical X
(more than 2 categories!)}
\protect\hypertarget{interpretation-of-hatbeta_1-with-categorical-x-more-than-2-categories-1}{}

\begin{itemize}
\tightlist
\item
  \textbf{Interpretation of \(\hat{\beta_1}\): 6184}

  \begin{itemize}
  \tightlist
  \item
    Generic (for categorical): the average effect of being {[}specific
    non-reference category{]} as opposed to {[}reference category{]} is
    associated with a \(\hat{\beta_1}\) change in Y
  \item
    Specific for this example: the average effect of being republican as
    opposed to being a democrat is associated with a \$6,184 increase in
    annual income
  \end{itemize}
\item
  We interpret \(\hat{\beta_2}\) and \(\hat{\beta_3}\) the same way!
\item
  \textbf{Interpretation of \(\hat{\beta_2}\): -1736}; the average
  effect of being independent as opposed to being a democrat is
  associated with a \$1,736 decrease in annual income (but not
  significant!)
\item
  \textbf{Interpretation of \(\hat{\beta_3}\): -3335}; the average
  effect of having an unknown political party as opposed to being a
  democrat is associated with a \$3,335 decrease in annual income (but
  not significant!)
\end{itemize}

\end{frame}

\begin{frame}{Prediction with Categorical X (more than 2 categories!)}
\protect\hypertarget{prediction-with-categorical-x-more-than-2-categories}{}

\begin{itemize}
\tightlist
\item
  Prediction with categorical X works the exact same way!
\item
  Show on Whiteboard

  \begin{itemize}
  \tightlist
  \item
    Population Model
  \item
    OLS Line with estimates
  \item
    Calculate predicted income for republicans, independents, democrats
  \end{itemize}
\end{itemize}

\end{frame}

\begin{frame}[fragile]{R Shortcut for Creating Dummies for Vars with 2+
categories}
\protect\hypertarget{r-shortcut-for-creating-dummies-for-vars-with-2-categories}{}

\begin{itemize}
\tightlist
\item
  It's a pain to create dummy versions for all categorical variables
  with 2+ categories

  \begin{itemize}
  \tightlist
  \item
    Some categorical variables can have many categories, which means you
    have to create as many dummy variables as you have categories (ugh!)
  \end{itemize}
\item
  There's an R shortcut!

  \begin{itemize}
  \tightlist
  \item
    Create one categorical variable with as many categories neeeded

    \begin{itemize}
    \tightlist
    \item
      we create it using \texttt{mutate()} + \texttt{case\_when()}
    \end{itemize}
  \item
    We insert the new categorical variable into our regression
  \item
    R ``creates'' the dummies for each category on the ``backend''
  \item
    R will assume lowest value category is reference; OR we can
    explicitly indicate what the reference category is!
  \end{itemize}
\end{itemize}

\end{frame}

\hypertarget{homework-data}{%
\section{Homework Data}\label{homework-data}}

\begin{frame}{Educational Longitudinal Study Data}
\protect\hypertarget{educational-longitudinal-study-data}{}

\begin{itemize}
\tightlist
\item
  Educational Longitudinal Study of 2002

  \begin{itemize}
  \tightlist
  \item
    (\url{https://nces.ed.gov/surveys/els2002/}){[}ELS Website{]}
  \item
    Nationally representative, longitudinal study of 10th graders in
    2002 and 12th graders in 2004
  \item
    Students followed throughout secondary and postsecondary years
  \item
    Surveys of students, their parents, math and English teachers, and
    school administrators
  \item
    Student assessments in math (10th \& 12th grades) and English (10th
    grade)
  \item
    High school transcripts available for research on coursetaking
  \end{itemize}
\end{itemize}

\medskip

\begin{itemize}
\tightlist
\item
  Problem Set \#7

  \begin{itemize}
  \tightlist
  \item
    You will need to download the data from D2L {[}all instructions are
    detailed in the assignment{]}
  \item
    I will give you all coded needed to create new variables; but try to
    get a bit of intuition behind the code {[}all based on what we
    learned this lecture{]}
  \end{itemize}
\end{itemize}

\end{frame}

\end{document}
