\PassOptionsToPackage{unicode=true}{hyperref} % options for packages loaded elsewhere
\PassOptionsToPackage{hyphens}{url}
\PassOptionsToPackage{dvipsnames,svgnames*,x11names*}{xcolor}
%
\documentclass[8pt,ignorenonframetext,dvipsnames]{beamer}
\usepackage{pgfpages}
\setbeamertemplate{caption}[numbered]
\setbeamertemplate{caption label separator}{: }
\setbeamercolor{caption name}{fg=normal text.fg}
\beamertemplatenavigationsymbolsempty
% Prevent slide breaks in the middle of a paragraph:
\widowpenalties 1 10000
\raggedbottom
\setbeamertemplate{part page}{
\centering
\begin{beamercolorbox}[sep=16pt,center]{part title}
  \usebeamerfont{part title}\insertpart\par
\end{beamercolorbox}
}
\setbeamertemplate{section page}{
\centering
\begin{beamercolorbox}[sep=12pt,center]{part title}
  \usebeamerfont{section title}\insertsection\par
\end{beamercolorbox}
}
\setbeamertemplate{subsection page}{
\centering
\begin{beamercolorbox}[sep=8pt,center]{part title}
  \usebeamerfont{subsection title}\insertsubsection\par
\end{beamercolorbox}
}
\AtBeginPart{
  \frame{\partpage}
}
\AtBeginSection{
  \ifbibliography
  \else
    \frame{\sectionpage}
  \fi
}
\AtBeginSubsection{
  \frame{\subsectionpage}
}
\usepackage{lmodern}
\usepackage{amssymb,amsmath}
\usepackage{ifxetex,ifluatex}
\usepackage{fixltx2e} % provides \textsubscript
\ifnum 0\ifxetex 1\fi\ifluatex 1\fi=0 % if pdftex
  \usepackage[T1]{fontenc}
  \usepackage[utf8]{inputenc}
  \usepackage{textcomp} % provides euro and other symbols
\else % if luatex or xelatex
  \usepackage{unicode-math}
  \defaultfontfeatures{Ligatures=TeX,Scale=MatchLowercase}
\fi
% use upquote if available, for straight quotes in verbatim environments
\IfFileExists{upquote.sty}{\usepackage{upquote}}{}
% use microtype if available
\IfFileExists{microtype.sty}{%
\usepackage[]{microtype}
\UseMicrotypeSet[protrusion]{basicmath} % disable protrusion for tt fonts
}{}
\IfFileExists{parskip.sty}{%
\usepackage{parskip}
}{% else
\setlength{\parindent}{0pt}
\setlength{\parskip}{6pt plus 2pt minus 1pt}
}
\usepackage{xcolor}
\usepackage{hyperref}
\hypersetup{
            pdftitle={Introduction to Multivariate Regression \& Program Evaluation},
            pdfauthor={Lecture 15},
            colorlinks=true,
            linkcolor=Maroon,
            filecolor=Maroon,
            citecolor=Blue,
            urlcolor=blue,
            breaklinks=true}
\urlstyle{same}  % don't use monospace font for urls
\newif\ifbibliography
\setlength{\emergencystretch}{3em}  % prevent overfull lines
\providecommand{\tightlist}{%
  \setlength{\itemsep}{0pt}\setlength{\parskip}{0pt}}
\setcounter{secnumdepth}{0}

% set default figure placement to htbp
\makeatletter
\def\fps@figure{htbp}
\makeatother


%packages
\usepackage{graphicx}
\usepackage{rotating}
\usepackage{hyperref}

\usepackage{tikz} % used for text highlighting, amongst others
\usepackage{comment}

%title slide stuff
%\institute{Department of Education}
%\title{Managing and Manipulating Data Using R}

%
\setbeamertemplate{navigation symbols}{} % get rid of navigation icons:
\setbeamertemplate{footline}[page number]

%\setbeamertemplate{frametitle}{\thesection \hspace{0.2cm} \insertframetitle}
\setbeamertemplate{section in toc}[sections numbered]
%\setbeamertemplate{subsection in toc}[subsections numbered]
\setbeamertemplate{subsection in toc}{%
  \leavevmode\leftskip=3.2em\color{gray}\rlap{\hskip-2em\inserttocsectionnumber.\inserttocsubsectionnumber}\inserttocsubsection\par
}

%define colors
%\definecolor{uva_orange}{RGB}{216,141,42} % UVa orange (Rotunda orange)
\definecolor{mygray}{rgb}{0.95, 0.95, 0.95} % for highlighted text
	% grey is equal parts red, green, blue. higher values >> lighter grey
	%\definecolor{lightgraybo}{rgb}{0.83, 0.83, 0.83}

% new commands

%highlight text with very light grey
\newcommand*{\hlg}[1]{%
	\tikz[baseline=(X.base)] \node[rectangle, fill=mygray] (X) {#1};%
}
%, inner sep=0.3mm
%highlight text with very light grey and use font associated with code
\newcommand*{\hlgc}[1]{\texttt{\hlg{#1}}}

%modifying back ticks to add grey background
\let\OldTexttt\texttt
\renewcommand{\texttt}[1]{\OldTexttt{\hlg{#1}}}


\begin{comment}

% Font
\usepackage[defaultfam,light,tabular,lining]{montserrat}
\usepackage[T1]{fontenc}
\renewcommand*\oldstylenums[1]{{\fontfamily{Montserrat-TOsF}\selectfont #1}}

% Change color of boldface text to darkgray
\renewcommand{\textbf}[1]{{\color{darkgray}\bfseries\fontfamily{Montserrat-TOsF}#1}}

% Bullet points
\setbeamertemplate{itemize item}{\color{BlueViolet}$\circ$}
\setbeamertemplate{itemize subitem}{\color{BrickRed}$\triangleright$}
\setbeamertemplate{itemize subsubitem}{$-$}

% Reduce space before lists
%\addtobeamertemplate{itemize/enumerate body begin}{}{\vspace*{-8pt}}

\let\olditem\item
\renewcommand{\item}{%
  \olditem\vspace{4pt}
}

% decreasing space before and after level-2 bullet block
%\addtobeamertemplate{itemize/enumerate subbody begin}{}{\vspace*{-3pt}}
%\addtobeamertemplate{itemize/enumerate subbody end}{}{\vspace*{-3pt}}

% decreasing space before and after level-3 bullet block
%\addtobeamertemplate{itemize/enumerate subsubbody begin}{}{\vspace*{-2pt}}
%\addtobeamertemplate{itemize/enumerate subsubbody end}{}{\vspace*{-2pt}}

%Section numbering
\setbeamertemplate{section page}{%
    \begingroup
        \begin{beamercolorbox}[sep=10pt,center,rounded=true,shadow=true]{section title}
        \usebeamerfont{section title}\thesection~\insertsection\par
        \end{beamercolorbox}
    \endgroup
}

\setbeamertemplate{subsection page}{%
    \begingroup
        \begin{beamercolorbox}[sep=6pt,center,rounded=true,shadow=true]{subsection title}
        \usebeamerfont{subsection title}\thesection.\thesubsection~\insertsubsection\par
        \end{beamercolorbox}
    \endgroup
}

\end{comment}

\title{Introduction to Multivariate Regression \& Program Evaluation}
\providecommand{\subtitle}[1]{}
\subtitle{HED 612}
\author{Lecture 15}
\date{}

\begin{document}
\frame{\titlepage}

\begin{frame}
\tableofcontents[hideallsubsections]
\end{frame}
\begin{frame}{Where are we going\ldots{}.}
\protect\hypertarget{where-are-we-going.}{}

\begin{itemize}
\tightlist
\item
  Our last Lecture!

  \begin{itemize}
  \tightlist
  \item
    End of Semester House keeping items
  \item
    Categorical by Categorical interactions
  \item
    Continuous by Continuous interactions
  \item
    Homework: Class Survey!
  \end{itemize}
\end{itemize}

\medskip

\begin{itemize}
\tightlist
\item
  Reading Day, ``No Class'' {[}5/7/2020{]}
\end{itemize}

\end{frame}

\begin{frame}[fragile]{\texttt{AER} R Package and Data!}
\protect\hypertarget{aer-r-package-and-data}{}

We're going to try out a textbook I'm considering for HED 613 that comes
with an accompanying R package

\begin{itemize}
\tightlist
\item
  Applied Econometrics with R, Christian Kleiber \& Achim Zeileis
\item
  \href{https://rdrr.io/cran/AER/f/inst/doc/AER.pdf}{\texttt{AER}
  Package}

  \begin{itemize}
  \tightlist
  \item
    Comes with different functions and datasets!
  \end{itemize}
\end{itemize}

\medskip

Current Population Survey

\begin{itemize}
\item
  The Current Population Survey (CPS), sponsored jointly by the U.S.
  Census Bureau and the U.S. Bureau of Labor Statistics (BLS), is the
  primary source of labor force statistics for the population of the
  United States
\item
  CORRECTION FROM LECTURE 14

  \begin{itemize}
  \tightlist
  \item
    \texttt{earnings} = average hourly earnings
  \item
    \texttt{earnings} \(\ne\) yearly income in \$000s
  \end{itemize}
\end{itemize}

\end{frame}

\hypertarget{end-of-semester-logistics}{%
\section{End of Semester Logistics}\label{end-of-semester-logistics}}

\begin{frame}{Grading}
\protect\hypertarget{grading}{}

\begin{itemize}
\tightlist
\item
  Submit whatever homework assignments you can by Friday, May 8th, 2010

  \begin{itemize}
  \tightlist
  \item
    Your course grade will be based on what you can submit!
  \item
    Don't stress trying to finish them all if you can't!
  \item
    But be sure to download all materials sometime soon after finals!
  \end{itemize}
\end{itemize}

\medskip

\begin{itemize}
\tightlist
\item
  For those pursuing final projects\ldots{}

  \begin{itemize}
  \tightlist
  \item
    I will absolutely consider final projects as ``ongoing'' if you'd
    like to keep working on them!
  \item
    If you would like detailed feedback, submit what you have by May 8th
  \item
    I am happy to work with you all through Summer and Fall 2020 as you
    continue working on these
  \end{itemize}
\end{itemize}

\medskip

\begin{itemize}
\tightlist
\item
  Next steps for those interested in quantitative research!

  \begin{itemize}
  \tightlist
  \item
    Continue taking classes in both statistics and data management!

    \begin{itemize}
    \tightlist
    \item
      Fall 2020: HED 696C Data Management and Manipulation in R
    \item
      Spring 2021: HED 613 Regression modeling with non-continuous
      dependent variables
    \item
      Classes in Sociology and other COE departments are good too!
    \end{itemize}
  \item
    Seek opportunities to practice these skills!
  \end{itemize}
\end{itemize}

\end{frame}

\begin{frame}{Student Course Survey (formerly Teacher Course
Evaluations)}
\protect\hypertarget{student-course-survey-formerly-teacher-course-evaluations}{}

\begin{itemize}
\tightlist
\item
  Student course survey's have been formally canceled!
\item
  But I would love your feedback (the good and the bad!)
\item
  PLEASE TAKE THIS SURVEY

  \begin{itemize}
  \tightlist
  \item
    It contains same questions as the student course surveys
  \item
    It is anonymous!
  \item
    Your honest feedback will help me improve the course!
  \end{itemize}
\end{itemize}

\end{frame}

\hypertarget{interaction-effects-continued}{%
\section{Interaction Effects
Continued\ldots{}}\label{interaction-effects-continued}}

\begin{frame}{What are interaction effects?}
\protect\hypertarget{what-are-interaction-effects}{}

\begin{itemize}
\tightlist
\item
  Simple hypothesis

  \begin{itemize}
  \tightlist
  \item
    X has an effect on Y
  \item
    Ex: Participation in MAS (X) has an effect on graduation (Y)
  \end{itemize}
\item
  Interactions = Conditional Hypothesis

  \begin{itemize}
  \tightlist
  \item
    The effect of X on Y depends on a third variable
  \item
    Ex: the effect of MAS participation (X) on graduation (Y) differs by
    race (Z)
  \end{itemize}
\item
  What is an interaction effect? (i.e., ``moderators'')

  \begin{itemize}
  \tightlist
  \item
    An interaction effect is when the relationship between two variables
    (X and Y) depends on the value of a third variable Z
  \end{itemize}
\end{itemize}

\end{frame}

\begin{frame}{Interaction Effects}
\protect\hypertarget{interaction-effects}{}

\begin{itemize}
\tightlist
\item
  Interaction Effects are difficult!

  \begin{itemize}
  \tightlist
  \item
    It takes a while (and lots of practice!) to get comfortable thinking
    about and interpreting interaction effects
  \end{itemize}
\item
  I will provide an ``introduction'' to interaction effects!

  \begin{itemize}
  \tightlist
  \item
    We will continue to learn interaction effects in HED 613
  \item
    If you don't take HED 613, seek out more opportunities to learn
    interaction effects!
  \item
    Read empirical pieces that use/interpret interaction effects!
  \end{itemize}
\end{itemize}

\end{frame}

\begin{frame}{Three cases of interaction effects}
\protect\hypertarget{three-cases-of-interaction-effects}{}

\textbf{1. Interaction between a categorical and continuous variable}
{[}last week{]}

\begin{itemize}
\tightlist
\item
  Example: the effect of years of schooling (X) on earnings (Y) differs
  for men and women (Z)
\end{itemize}

\textbf{2. Interaction between two categorical variables} {[}today{]}

\begin{itemize}
\tightlist
\item
  Example: the effect of having more than a high school diploma (X) on
  earnings (Y) differs for men and women (Z)
\end{itemize}

\textbf{3. Interaction between two continuous variables} {[}today{]}

\begin{itemize}
\tightlist
\item
  Example: the effect of years of schooling (X) on earnings (Y) differs
  by age (Z)
\end{itemize}

\end{frame}

\hypertarget{interactions-between-two-categorical-variables}{%
\section{Interactions between two categorical
variables}\label{interactions-between-two-categorical-variables}}

\begin{frame}{Interaction between two categorical variables}
\protect\hypertarget{interaction-between-two-categorical-variables}{}

\textbf{RQ: Does the effect of having more than a HS diploma (X) on
earnings (Y) differ by gender (Z)?}

\begin{itemize}
\tightlist
\item
  Categorical by Categorical Interaction

  \begin{itemize}
  \tightlist
  \item
    Y = earnings
  \item
    X = 0/1 more than a HS diploma
  \item
    Z (interaction variable) = 0/1 Women (0=men)
  \end{itemize}
\item
  Simple hypothesis

  \begin{itemize}
  \tightlist
  \item
    Having more than a HS diploma affects earnings
  \end{itemize}
\item
  Conditional hypothesis (interaction effect)

  \begin{itemize}
  \tightlist
  \item
    The effect of having more than a HS diploma (X) on earnings (Y)
    differs for women and men
  \end{itemize}
\end{itemize}

\end{frame}

\begin{frame}{Simple Regression (no interaction effect)}
\protect\hypertarget{simple-regression-no-interaction-effect}{}

\begin{itemize}
\tightlist
\item
  Simple Regression

  \begin{itemize}
  \tightlist
  \item
    What is the having more than a HS diploma on earnings?
  \item
    Population regression model:

    \begin{itemize}
    \tightlist
    \item
      \(Y_i = \beta_0 + \beta_1X_{1i} + u_i\)
    \item
      where Y= earnings, \(X_{1}\) = 0/1 having more than a HS diploma
    \end{itemize}
  \end{itemize}
\item
  Run model in R
\item
  \(\hat{earnings} = \hat{\beta_0} + \hat{\beta_1}*morehs_{i}\)

  \begin{itemize}
  \tightlist
  \item
    \(\hat{earnings} = 14.48726 + 6.54299*morehs_{i}\)
  \end{itemize}
\item
  \(\hat{\beta_1}\) = 6.54299

  \begin{itemize}
  \tightlist
  \item
    On average, having more than a high school diploma as opposed to
    having a high school diploma or less is associated with a \$6.54
    increase in hourly earnings
  \item
    \(\hat{\beta_1}\) is significant at the 0.000 level
  \end{itemize}
\end{itemize}

\end{frame}

\begin{frame}{Multivariate Regression (no interaction effect)}
\protect\hypertarget{multivariate-regression-no-interaction-effect}{}

\begin{itemize}
\tightlist
\item
  Multivariate Regression

  \begin{itemize}
  \tightlist
  \item
    What is the effect of having more than a HS diploma on earnings
    controlling for gender?
  \item
    Population regression model:

    \begin{itemize}
    \tightlist
    \item
      \(Y_i = \beta_0 + \beta_1X_{1i} + \beta_2X_{2i} + u_i\)
    \item
      where Y= earnings, \(X_{1}\) = 0/1 more than HS, \(X_{2}\) = 0/1
      women
    \end{itemize}
  \end{itemize}
\item
  Run model in R
\item
  \(\hat{Y_i} = \hat{\beta_0} + \hat{\beta_1}X_{1i} + \hat{\beta_2}X_{2i}\)

  \begin{itemize}
  \tightlist
  \item
    \(\hat{Y_i} = 16.16394 + 6.81100*X_{1i} - 4.17302*X_{2i}\)
  \end{itemize}
\item
  \(\hat{\beta_1}\) = 6.81100

  \begin{itemize}
  \tightlist
  \item
    On average, having more than a high school diploma as opposed to
    having a high school diploma or less is associated with a \$6.81
    increase in hourly earnings, controlling for gender
  \item
    \(\hat{\beta_1}\) is significant at the 0.000 level
  \end{itemize}
\item
  \(\hat{\beta_2}\) = -4.17302

  \begin{itemize}
  \tightlist
  \item
    On average, identifying as a woman as opposed to identifying as a
    man, is associated with a \$4.17 decrease in hourly earnings,
    controlling for whether or not participant has more than a HS
    diploma
  \item
    \(\hat{\beta_2}\) is significant at the 0.000 level
  \end{itemize}
\end{itemize}

\end{frame}

\begin{frame}{Multivariate Regression (no interaction effect)}
\protect\hypertarget{multivariate-regression-no-interaction-effect-1}{}

\begin{itemize}
\tightlist
\item
  The multivariate regression which only controls for gender\ldots{}

  \begin{itemize}
  \tightlist
  \item
    Assumes that the effect of having more than a HS diploma is the same
    for men and women!
  \item
    We can extend the model by allowing the effect of having more than a
    high school diploma to depend on gender by including an interaction
    between these two variables!
  \end{itemize}
\item
  Same steps as last week\ldots{} {[}refresher!{]}

  \begin{itemize}
  \tightlist
  \item
    We will run separate models by different groups of Z
  \item
    Then we will run interaction model
  \end{itemize}
\end{itemize}

\end{frame}

\begin{frame}{Separate Models for each group of Z}
\protect\hypertarget{separate-models-for-each-group-of-z}{}

\begin{itemize}
\tightlist
\item
  What is the effect having more than a HS Diploma on earnings for
  women?

  \begin{itemize}
  \tightlist
  \item
    Run model in R
  \item
    \(\hat{Y_i} = 12.31394 + 6.30489*X_{1i}\)
  \item
    \(\hat{\beta_0}\): predicted earnings for women with a HS diploma or
    less
  \item
    \(\hat{\beta_1}\): On average, having more than a HS diploma as
    opposed to having a HS diploma or less is associated with a \$6.31
    increase in hourly earnings \emph{for women}
  \item
    Y\textbar{} X=1: \(12.31394 + 6.30489*1\) = \$18.61883 hourly
    earnings
  \end{itemize}
\item
  What is the effect having more than a HS Diploma on earnings for men?

  \begin{itemize}
  \tightlist
  \item
    Run model in R
  \item
    \(\hat{Y_i} = 15.94698 + 7.18773*X_{1i}\)
  \item
    \(\hat{\beta_0}\): predicted earnings for men with a HS diploma or
    less
  \item
    \(\hat{\beta_1}\): On average, having more than a HS diploma as
    opposed to having a HS diploma or less is associated with a \$7.19
    increase in hourly earnings \emph{for men}
  \item
    Y\textbar{} X=1: \(15.94698 + 7.18773*1\) = \$23.13471 hourly
    earnings
  \end{itemize}
\end{itemize}

\end{frame}

\begin{frame}{Interaction Model for Two Categorical Variables}
\protect\hypertarget{interaction-model-for-two-categorical-variables}{}

\begin{itemize}
\tightlist
\item
  Let's run an interaction effect model to investigate whether the
  effect of having more than a HS diploma (X) on earnings (Y) differs by
  gender (Z)
\item
  Population regression model

  \begin{itemize}
  \tightlist
  \item
    \(Y_i = \beta_0 + \beta_1X_{1i} + \beta_2Z_{i} + \beta_3(X_{1i}*Z_{i}) + u_i\)

    \begin{itemize}
    \tightlist
    \item
      where Y= earnings, \(X_{1}\) = 0/1 More than HS, \(Z_{i}\) = 0/1
      Women
    \item
      \(X_{1i}*Z_{i}\) = interaction for more than HS and gender
    \end{itemize}
  \end{itemize}
\end{itemize}

\medskip

\begin{itemize}
\tightlist
\item
  OLS Prediction line without estimates

  \begin{itemize}
  \tightlist
  \item
    \(\hat{Y_i} = \hat{\beta_0} + \hat{\beta_1}X_{1i} + \hat{\beta_2}Z_{i} + \hat{\beta_3}(X_{1i}*Z_{i})\)
  \end{itemize}
\end{itemize}

\medskip

\begin{itemize}
\tightlist
\item
  \textbf{\(\hat{\beta_0}\) = \(\hat{Y_i}\) when X1=0 and Z=0}

  \begin{itemize}
  \tightlist
  \item
    Predicted earnings for men (Z=0) with a high school diploma or less
    (X1=0)
  \end{itemize}
\item
  \textbf{\(\hat{\beta_1}\) = change in \(\hat{Y_i}\) for X1=1 as
  opposed to X1=0, when Z=0}

  \begin{itemize}
  \tightlist
  \item
    Change in earnings for having more than a HS diploma (X1=1\_) as
    opposed to having a HS diploma or less (X1=0) for men (Z=0)
  \end{itemize}
\item
  \textbf{\(\hat{\beta_2}\) = change in \(\hat{Y_i}\) for Z1=1 as
  opposed to Z1=0, when X1=0}

  \begin{itemize}
  \tightlist
  \item
    Change in earnings for women (Z=1) as opposed to men (Z=0) with a HS
    diploma or less (X1=0)
  \end{itemize}
\item
  \textbf{\(\hat{\beta_3}\) = interaction term: how much the effect of
  X1 on \(\hat{Y_i}\) changes when Z ``increases by one unit'' or when
  Z=1 as opposed to Z=0}

  \begin{itemize}
  \tightlist
  \item
    change in the effect of having more than a HS diploma on earnings
    for for Z=1 as opposed to Z=0
  \end{itemize}
\end{itemize}

\end{frame}

\begin{frame}{Interaction Model for Categorical by Categorical
Interaction}
\protect\hypertarget{interaction-model-for-categorical-by-categorical-interaction}{}

\begin{itemize}
\tightlist
\item
  Run in R
\item
  \(\hat{Y_i} = \hat{\beta_0} + \hat{\beta_1}X_{1i} + \hat{\beta_2}Z_{i} + \hat{\beta_3}(X_{1i}*Z_{i})\)
\item
  \(\hat{Y_i} = 15.94698 + 7.18773*X_{1i} - 3.63304*Z_{i} - 0.88283*(X_{1i}*Z_{i})\)
\end{itemize}

\end{frame}

\begin{frame}{What do we want to know from interactions?}
\protect\hypertarget{what-do-we-want-to-know-from-interactions}{}

\begin{enumerate}
\item
  Is there an interaction effect?
\item
  What is the predicted value of Y for Z=0 and Z=1 at different values
  of X?
\item
  What is the effect of X on Y for different values of Z?
\end{enumerate}

\end{frame}

\begin{frame}{Is there an interaction effect?}
\protect\hypertarget{is-there-an-interaction-effect}{}

\begin{itemize}
\tightlist
\item
  Is the effect of years having more than a high school degree
  significantly different for women vs men?

  \begin{itemize}
  \tightlist
  \item
    \(Y_i = \beta_0 + \beta_1X_{1i} + \beta_2Z_{i} + \beta_3(X_{1i}*Z_{i}) + u_i\)
  \end{itemize}
\item
  Hypothesis test

  \begin{itemize}
  \tightlist
  \item
    \(H_0: \beta_3 =0\) vs \(H_0: \beta_3 \ne 0\)
  \item
    In other words, test whether the beta coefficient for the
    interaction term is significantly different from zero. If so, then
    there is an interaction!
  \end{itemize}
\item
  \(\hat{\beta_3}\) = -0.88283 , p-value of 0.000

  \begin{itemize}
  \tightlist
  \item
    We reject \(H_0\) at \(\alpha\) of 0.05
  \item
    There is a statistically significant interaction!
  \end{itemize}
\item
  Magnitude

  \begin{itemize}
  \tightlist
  \item
    If \(\hat{\beta_3}\) is greater than zero (and statistically
    significant!) the effect of X on Y is larger when Z=1 than when Z=0
  \item
    If \(\hat{\beta_3}\) is less than zero (and statistically
    significant!) the effect of X on Y is smaller when Z=1 than when Z=0
  \end{itemize}
\end{itemize}

\end{frame}

\begin{frame}{What is the predicted value of Y for Z=0 and Z=1 at
different values of X}
\protect\hypertarget{what-is-the-predicted-value-of-y-for-z0-and-z1-at-different-values-of-x}{}

\begin{itemize}
\tightlist
\item
  What is the predicted earnings (Y) for men (Z=0) with more than a HS
  diploma (X=1)?

  \begin{itemize}
  \tightlist
  \item
    \(\hat{Y_i} = 15.94698 + 7.18773*X_{1i} - 3.63304*Z_{i} - 0.88283*(X_{1i}*Z_{i})\)
  \item
    Z=0 \& X=1:
    \(15.94698 + (7.18773*1) - (3.63304*0) - (0.88283*(1*0))\)
  \item
    \(\$23.13 = 15.94698 + (7.18773)\)
  \end{itemize}
\item
  What is the predicted earnings (Y) for women (Z=1) with more than a HS
  diploma (X=1)?

  \begin{itemize}
  \tightlist
  \item
    \(\hat{Y_i} = 15.94698 + 7.18773*X_{1i} - 3.63304*Z_{i} - 0.88283*(X_{1i}*Z_{i})\)
  \item
    Z=1 \& X=1:
    \(15.94698 + (7.18773*1) - (3.63304*1) - (0.88283*(1*1))\)
  \item
    \(\$18.62 = 15.94698 + (7.18773) - (3.63304) - (0.88283)\)
  \end{itemize}
\end{itemize}

\medskip

\begin{itemize}
\tightlist
\item
  Same result as when we ran model separately by sample!

  \begin{itemize}
  \tightlist
  \item
    Men: \(15.94698 + 7.18773*1\) = 23.13471
  \item
    Women: \(12.31394 + 6.30489*1\) = 18.61883
  \end{itemize}
\item
  But running separately by sample does not test whether there is a
  statistically significant interaction!
\end{itemize}

\end{frame}

\begin{frame}{What is the effect of X on Y for different values of Z}
\protect\hypertarget{what-is-the-effect-of-x-on-y-for-different-values-of-z}{}

\begin{itemize}
\tightlist
\item
  \(\hat{Y_i} = \hat{\beta_0} + \hat{\beta_1}X_{1i} + \hat{\beta_2}*1 + \hat{\beta_3}(X_{1i}*1)\)
\item
  \(\hat{Y_i} = 15.94698 + 7.18773*X_{1i} - 3.63304*Z_{i} - 0.88283*(X_{1i}*Z_{i})\)

  \begin{itemize}
  \tightlist
  \item
    Remember that:
  \item
    \(\hat{\beta_1}\) = change in \(\hat{Y_i}\) for a one-unit increase
    in X1, when Z=0 -- \(\hat{\beta_1}\) = how much effect of X1 on Y
    changes when Z increases by one unit
  \end{itemize}
\end{itemize}

\medskip

\begin{itemize}
\tightlist
\item
  \textbf{\(\hat{\beta_1}\): Average effect of X on Y when Z=0}

  \begin{itemize}
  \tightlist
  \item
    \(\hat{\beta_1}\) = 7.18773
  \item
    change in \(\hat{Y_i}\) for X1=1 as opposed to X1=0, when Z=0\_\_
  \item
    Change in earnings for having more than a HS diploma (X1=1\_) as
    opposed to having a HS diploma or less (X1=0) for men (Z=0)
  \item
    On average, having more than a HS diploma (X1=1) as opposed to
    having a HS diploma or less (X1=0) is associated with \$7.19
    increase in hourly earnings for men (Z=0)
  \end{itemize}
\end{itemize}

\medskip

\begin{itemize}
\tightlist
\item
  \textbf{\((\hat{\beta_1} + \hat{\beta_3})\): Average effect of X on Y
  when Z=1}

  \begin{itemize}
  \tightlist
  \item
    \(\hat{\beta_1}\) = 7.18773; \(\hat{\beta_3}\) = - 0.88283
  \item
    7.18773 + - 0.88283 = 6.3049
  \item
    On average, having more than a HS diploma (X1=1) as opposed to
    having a HS diploma or less (X1=0) is associated with \$6.30
    increase in hourly earnings for women (Z=1)
  \end{itemize}
\end{itemize}

\medskip

\begin{itemize}
\tightlist
\item
  Same result as when we ran model separately by sample!
\end{itemize}

\end{frame}

\hypertarget{interactions-by-two-continuous-variables}{%
\section{Interactions by Two Continuous
Variables}\label{interactions-by-two-continuous-variables}}

\begin{frame}{Interactions by Two Continuous Variables}
\protect\hypertarget{interactions-by-two-continuous-variables-1}{}

\textbf{RQ: Does the effect of years of schooling (X) on earnings (Y)
differ by age (Z)?}

\begin{itemize}
\tightlist
\item
  Continuous by Categorical Interaction

  \begin{itemize}
  \tightlist
  \item
    Y = earnings
  \item
    X = years of schooling
  \item
    Z (interaction variable) = age
  \end{itemize}
\item
  Simple hypothesis

  \begin{itemize}
  \tightlist
  \item
    Years of schooling affects earnings
  \end{itemize}
\item
  Conditional hypothesis (interaction effect)

  \begin{itemize}
  \tightlist
  \item
    The effect of years of schooling (X) on earnings (Y) depends on age
  \end{itemize}
\end{itemize}

\end{frame}

\begin{frame}{Simple and Multivariate Regression (no interaction
effect)}
\protect\hypertarget{simple-and-multivariate-regression-no-interaction-effect}{}

Simple Regression

\begin{itemize}
\tightlist
\item
  Population Regression Model

  \begin{itemize}
  \tightlist
  \item
    \(Y_i = \beta_0 + \beta_1X_{1i} + u_i\)
  \item
    Where Y= earnings, \(X_1\) = years of schooling
  \end{itemize}
\item
  Run in R
\item
  OLS Prediction line with estimates

  \begin{itemize}
  \tightlist
  \item
    \(Y_i = -5.37626 + 1.74515*X_{1i}\)
  \item
    On average, a one-year increase in years of schooling is associated
    with a \$1.75 increase in hourly earnings
  \end{itemize}
\end{itemize}

\medskip

Multivariate Regression

\begin{itemize}
\tightlist
\item
  Population Regression Model

  \begin{itemize}
  \tightlist
  \item
    \(Y_i = \beta_0 + \beta_1X_{1i} + \beta_2X_{2i} + u_i\)
  \item
    Where Y= earnings, \(X_1\) = years of schooling, \(X_2\) = age
  \end{itemize}
\item
  Run in R
\item
  OLS Prediction line with estimates

  \begin{itemize}
  \tightlist
  \item
    \(Y_i = -11.177774 + 1.706443*X_{1i} + 0.153515*X_{2i}\)
  \item
    On average, a one-year increase in years of schooling is associated
    with a \$171 increase in hourly earnings, holding age constant
  \end{itemize}
\item
  But this assumes the effect of years of schooling is the same across
  all ages!
\end{itemize}

\end{frame}

\begin{frame}{Interaction Model for Two Categorical Variables}
\protect\hypertarget{interaction-model-for-two-categorical-variables-1}{}

\begin{itemize}
\tightlist
\item
  Let's run an interaction effect model to investigate whether the
  effect of years of school (X) on earnings (Y) differs by age (Z)
\item
  Population regression model

  \begin{itemize}
  \tightlist
  \item
    \(Y_i = \beta_0 + \beta_1X_{1i} + \beta_2Z_{i} + \beta_3(X_{1i}*Z_{i}) + u_i\)

    \begin{itemize}
    \tightlist
    \item
      where Y= earnings, \(X_{1}\) = years of school, \(Z_{i}\) = age
    \item
      \(X_{1i}*Z_{i}\) = interaction for years of schooling and age
    \end{itemize}
  \end{itemize}
\end{itemize}

\medskip

\begin{itemize}
\tightlist
\item
  OLS Prediction line without estimates

  \begin{itemize}
  \tightlist
  \item
    \(\hat{Y_i} = \hat{\beta_0} + \hat{\beta_1}X_{1i} + \hat{\beta_2}Z_{i} + \hat{\beta_3}(X_{1i}*Z_{i})\)
  \end{itemize}
\end{itemize}

\medskip

\begin{itemize}
\tightlist
\item
  \textbf{\(\hat{\beta_0}\) = \(\hat{Y_i}\) when X1=0 and Z=0}

  \begin{itemize}
  \tightlist
  \item
    Predicted earnings for observations with zero years of schooling
    (X=0) and age zero (Z=0)
  \end{itemize}
\item
  \textbf{\(\hat{\beta_1}\) = change in \(\hat{Y_i}\) one-unit increase
  in X1, when Z=0}

  \begin{itemize}
  \tightlist
  \item
    Change in earnings for one year increase in schooling when age is
    zero
  \end{itemize}
\item
  \textbf{\(\hat{\beta_2}\) = change in \(\hat{Y_i}\) for one-unit
  increase in Z, when X1=0}

  \begin{itemize}
  \tightlist
  \item
    Change in earnings for one year increase in age when years of
    schooling is zero
  \end{itemize}
\item
  \textbf{\(\hat{\beta_3}\) = interaction term: how much the effect of
  X1 on \(\hat{Y_i}\) changes when Z increases by one unit}

  \begin{itemize}
  \tightlist
  \item
    change in the effect of years of schooling for a one-year increase
    in age
  \end{itemize}
\end{itemize}

\end{frame}

\begin{frame}{Interaction Model for Continuous by Continuous
Interaction}
\protect\hypertarget{interaction-model-for-continuous-by-continuous-interaction}{}

\begin{itemize}
\tightlist
\item
  Run in R
\item
  \(\hat{Y_i} = \hat{\beta_0} + \hat{\beta_1}X_{1i} + \hat{\beta_2}Z_{i} + \hat{\beta_3}(X_{1i}*Z_{i})\)
\item
  \(\hat{Y_i} = -7.032907 + 1.398881*X_{1i} + 0.055153*Z_{i} + 0.007281*(X_{1i}*Z_{i})\)
\end{itemize}

\medskip

\begin{itemize}
\tightlist
\item
  \textbf{Is there an interaction effect?}
\item
  \(\beta_3\) is significant at the 0.000

  \begin{itemize}
  \tightlist
  \item
    Yes, there is a statistically significant interaction between years
    of schooling and age!
  \end{itemize}
\item
  Positive \(\beta_3\) coefficient means that the effect of years of
  schooling gets stronger as age increases!
\item
  Negative \(\beta_3\) coefficient means that the effect of years of
  schooling gets weaker as age increases!
\end{itemize}

\end{frame}

\begin{frame}{What is the predicted value of Y for different values of X
and Z}
\protect\hypertarget{what-is-the-predicted-value-of-y-for-different-values-of-x-and-z}{}

\begin{itemize}
\tightlist
\item
  X = 16 years of schooling (BA) and Z= 24 years old

  \begin{itemize}
  \tightlist
  \item
    Y \textbar{}X=16, Z=24:
    \(\hat{Y_i} = -7.032907 + (1.398881*16) + (0.055153*24) + (0.007281*(16*24))\)
  \item
    \(\hat{Y_i} = -7.032907 + 22.3821 + 1.323672 + 0.007281*(384)\)
  \item
    \(19.46877 = -7.032907 + 22.3821 + 1.323672 + 2.795904\)
  \end{itemize}
\end{itemize}

\medskip

\begin{itemize}
\tightlist
\item
  X = 16 years of schooling (BA) and Z = 45 years old

  \begin{itemize}
  \tightlist
  \item
    Y \textbar{}X=16, Z=45:
    \(\hat{Y_i} = -7.032907 + (1.398881*16) + (0.055153*45) + (0.007281*(16*45))\)
  \item
    \(\hat{Y_i} = -7.032907 + 22.3821 + 2.481885+ 0.007281*(720)\)
  \item
    \(21.91519 = -7.032907 + 22.3821 + 1.323672 + 5.24232\)
  \end{itemize}
\end{itemize}

\end{frame}

\hypertarget{thank-you-and-take-care-of-yourselves}{%
\section{THANK YOU AND TAKE CARE OF
YOURSELVES}\label{thank-you-and-take-care-of-yourselves}}

\end{document}
